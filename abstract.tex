\begin{foreignabstract}

% Objetivo
	% O que foi pesquisado
	% O porquê foi pesquisado
		%	Justificativa
		%	Originalidade
		%	Lógica que guiou a investigação
		
This work presents a method to predict the failure of a robotic manipulator's
task due to the flexibility of the elastic bodies that serves as a base for the
robot. The method also provides a tool to verify how modifications in the base
design can change the dynamic behaviour of the coupled system. The errors can be
the position, velocity and orientation of the tool attached to the tip of the
robot.
The motivation of such analysis is the crescent employment of service robots for
\textit{in situ} applications, where the use of a light and slender structure
for the base may be the best and feasible solution for placing the robot in the
service environment. The analysis are divided as: model the Dynamic Multibody
Systems of the robot, the base and the coupled base-robot, by Kane's method;
find the stiffness matrix of the base, by Finite Element Analysis; find the
Rayleigh parameters of the proportional damping matrix of the base, by
Experimental Modal Analysis.
EMMA, an \textit{in situ} robotic hardcoating system is chosen as the case of
study for simulations. The simulations are done for two different trajectories
and three different bases. The results of the reference model (rigid base) and
the coupled model (robot - flexible base) are then compared and shows the
changes in the controled parameters due to the flexibility of the base. This
analysis allows the verification of compliance to the hardcoating requirements
like position, velocity and orientation of the gun with respect to the surface.

% Métodos

% Resultados

% Conclusões


\end{foreignabstract}

