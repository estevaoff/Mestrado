\appendix
\chapter{Código Sophia-Maple}\label{apendiceA}

\section{Parâmetros do robô}

\lstset{% general command to set parameter(s)
basicstyle=\small,          % print whole listing small
keywordstyle=\color{black}\bfseries\underbar, % underlined bold black keywords
identifierstyle= ,          % nothing happens
commentstyle=\color{white}, % white comments
stringstyle=\ttfamily,      % typewriter type for strings
showstringspaces=false}     % no special string spaces

\begin{lstlisting}         
param_mh12:= {
mZ= 40.5,
mS= 36.7,
mL= 14.7,
mU= 25.3,
mR= 10.1,
mB= 18.6,
InZ11= 0.732, InZ22= 0.335, InZ33= 0.567, 
InZ12= 0.013, InZ13= 0.000, InZ23= 0.000,
InS11= 0.811, InS22= 0.695, InS33= 0.923, 
InS12= 0.293, InS13= 0.091, InS23= 0.064,
InL11= 0.051, InL22= 0.791, InL33= 0.798, 
InL12= -0.008, InL13= -0.032, InL23= 0.011,
InU11= 0.314, InU22= 0.367, InU33= 0.418, 
InU12= 0.149, InU13= -0.077, InU23= -0.057,
InR11= 0.173, InR22= 0.052, InR33= 0.148, 
InR12= -0.014, InR13= 0.001, InR23= 0.004,
InB11= 0.235, InB22= 0.039, InB33= 0.243, 
InB12= -0.018, InB13= 0.000, InB23= 0.000,
pcmZx= 0.079, pcmZy= -0.050, pcmZz= 0,
pZSx= 0.165, pZSy= 0, pZSz= 0,
pcmSx= 0.124, pcmSy= 0.027, pcmSz= 0.008,
pSLx= 0.285, pSLy= 0.155, pSLz= 0,
pcmLx= 0.281, pcmLy= -0.025, pcmLz= -0.115,
pLUx= 0.614, pLUy= 0, pLUz= 0,
pcmUx= 0.123, pcmUy= 0.115, pcmUz= -0.025,
pURx= 0.200, pURy= 0.640, pURz= 0,
pcmRx= 0.039, pcmRy= -0.224, pcmRz= 0,
pRBx= 0, pRBy= 0, pRBz=0,
pcmBx= 0, pcmBy= 0.146, pcmBz= 0,
pBTx = 0, pBTy= 0.442, pBTz=0,
kpS= 2000, tiS= 8000, tdS= 1000,
kpL= 4000, tiL= 8000, tdL= 1000,
kpU= 4000, tiU= 8000, tdU= 1000,
kpR= 2000, tiR= 1000, tdR= 150,
kpB= 2000, tiB= 1000, tdB= 150
};
\end{lstlisting}

\section{Modelo robô em base flexível}

\begin{lstlisting}
# Inicio das ferramentas do toolbox e do Sophia
restart;
with(plottools):
with(plots):
with(linalg):
with(CodeTools):
with(codegen, cost, optimize, makeproc):
read "C:\\Sophia\\sophia_V18";

# Numero de graus de liberdade do sistema:
gdl:= 11;
&kde gdl;

# Definicao dos referenciais e corpos do sistema
chainSimpRot([F, Th1, 1, q4],[Th1, Th2, 2, q5],[Th2, Z, 3, q6], 
[Z, S, 1, q7], [S, L, 3, q8], [L, U, 3, q9], [U, R, 2, q10], 
[R, B, 3, q11]):



\end{lstlisting}


