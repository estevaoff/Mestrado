\chapter{Introdução}

% Segundo a norma de formata{\c c}\~ao de teses e disserta{\c c}\~oes do
% Instituto Alberto Luiz Coimbra de P\'os-gradua{\c c}\~ao e Pesquisa de
% Engenharia (COPPE), toda abreviatura deve ser definida antes de
% utilizada.\abbrev{COPPE}{Instituto Alberto Luiz Coimbra de P\'os-gradua{\c
% c}\~ao e Pesquisa de Engenharia}
% 
% Do mesmo modo, \'e imprescind\'ivel definir os s\'imbolos, tal como o
% conjunto dos n\'umeros reais $\mathbb{R}$ e o conjunto vazio $\emptyset$.
% \symbl{$\mathbb{R}$}{Conjunto dos n\'umeros reais}
% \symbl{$\emptyset$}{Conjunto vazio}

% -.~.-.~.-.~.-.~.-.~.-.~.-.~.-.~.-.~.-.~.-.~.-
\section{Motivação}


% -.~.-.~.-.~.-.~.-.~.-.~.-.~.-.~.-.~.-.~.-.~.-
\section{Objetivos}

A ação dinâmica do robô ao realizar determinada tarefa irá originar, por reação,
deslocamentos dinâmicos em sua base. Estes deslocamentos causarão algum impacto
na tarefa a ser realizada como mudanças de orientação, velocidade e
posicionamento da ferramenta acoplada ao manipulador. Estes efeitos não são
considerados nos métodos de controle convencionais das juntas do manipulador,
porque não podem ser previstos sem que haja alguma instrumentação periférica,
capaz de medir e enviar as informações ao controlador.

Este trabalho tem como objetivo oferecer um método para calcular e quantificar
os deslocamentos teóricos da base, e assim calcular os erros de posicionamento,
velocidade e orientação no seu efetuador. Isto é feito a partir da
quebra do paradigma da base como uma estrutura idealmente rígida -- ou seja, de
rigidez tão grande que pudessem ser desconsiderados quaisquer efeitos da ação do
manipulador -- e considerando-a como um corpo elástico, com flexibilidade e
amortecimento.

Quantificando-se o comportamento dinâmico do sistema base e robô, pode-se então
avaliar os desvios causados na trajetória. Se o resultado demonstrar erros fora
de uma margem de tolerância admitida, o método pode ser utlizado iterativamente,
alterando-se parâmetros da base.
Após mudanças estratégicas destes parâmetros -- como materiais, configuração
geométrica, apoios, contravetamentos, massa, etc. -- obtém-se novos resultados.
O objetivo deste processo é, em um ambiente de simulação, verificar a nova base,
e a partir dos resultados fazer as modificações até atingir a tolerância que
satisfaz a qualidade da tarefa.

Além do método iterativo, tem-se por objetivo discutir estratégias de
instrumentação e controle, de forma a aperfeiçoar o método original, para levar
em consideração as perturbações causadas pela base e corrigir as diferenças de
posicionamento, velocidade e orientação das juntas do manipulador.

O método proposto tem como preocupação oferecer uma metodologia que permite ao
usuário utilizar as ferramentas que estão à sua disposição pessoal ou da
instituição (laboratório, empresa, universidade, etc.), não limitando ao uso de
qualquer \textit{software} específico. Apesar de terem sido utlilizados
programas de alto nível como Maple, Autodesk Simulation e SolidWorks, ficará
claro que diversos \textit{softwares} de Álgebra Computacional Simbólica (CAS),
Análise de Elementos Finitos (FEA) e Desenho Assistido por Computador (CAD),
respectivamente, poderiam ter sido utilizados.


% -.~.-.~.-.~.-.~.-.~.-.~.-.~.-.~.-.~.-.~.-.~.-
\section{Ferramentas utilizadas}

\subsubsection{Sophia-Maple}

\subsubsection{SolidWorks}

\subsubsection{Autodesk Simulation}

\subsubsection{SignalExpress}

\subsubsection{ME`Scope}

\subsubsection{\LaTeX\ e Eclipse}

\section{Notação}
