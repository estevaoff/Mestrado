\chapter{Introdução}

% Segundo a norma de formata{\c c}\~ao de teses e disserta{\c c}\~oes do
% Instituto Alberto Luiz Coimbra de P\'os-gradua{\c c}\~ao e Pesquisa de
% Engenharia (COPPE), toda abreviatura deve ser definida antes de
% utilizada.\abbrev{COPPE}{Instituto Alberto Luiz Coimbra de P\'os-gradua{\c
% c}\~ao e Pesquisa de Engenharia}
% 
% Do mesmo modo, \'e imprescind\'ivel definir os s\'imbolos, tal como o
% conjunto dos n\'umeros reais $\mathbb{R}$ e o conjunto vazio $\emptyset$.
% \symbl{$\mathbb{R}$}{Conjunto dos n\'umeros reais}
% \symbl{$\emptyset$}{Conjunto vazio}

% -.~.-.~.-.~.-.~.-.~.-.~.-.~.-.~.-.~.-.~.-.~.-
\section{Motivação}

Com uma população de robôs industriais cada vez maior em diversas áreas de
automação, surgem também novas ideias e aplicações. Seja para melhoria da
eficiência, impossibilidade de realização humana ou tarefas em ambientes hostis,
utilizar robôs se torna uma alternativa atrativa também para trabalhos
\textit{in situ}, ou operacional em campo, locais não planejados originalmente
para a aplicação robótica.
Ambientes bem estruturados são geralmente um requisito da maioria das aplicações
utilizando manipuladores robóticos, que não têm capacidade para lidar com ações
ou efeitos do ambiente e, caso ocorram, não pode-se garantir a segurança ou
qualidade da tarefa a ser executada.

Um exemplo de aplicação robótica \textit{in situ} é o sistema de manutenção de
pás de turbinas, denominado EMMA\cite{freitas2017state}.
É um projeto desenvolvido pela empresa THIRTEEN ROBOTICS em parceria com o
LEAD/COPPE-UFRJ e a Energia Sustentável do Brasil (ESBR). Este sistema consiste
em realizar um procedimento de manutenção do revestimento das pás de turbinas do
tipo Kaplan em usinas hidrelétricas, que sofrem desgaste por abrasão e
cavitação, perdendo sua camada de proteção devendo ser reaplicada. Atualmente,
as pás são desmontadas e levadas para um galpão em que, num ambiente bem
estruturado, um robô de grande porte reaplica o revestimento.
O objetivo do projeto EMMA é poupar todo o procedimento de desmontagem e
transporte destas pás, que geram um alto custo e tempo, por um sistema de
revestimento robótico \textit{in situ}, ou seja, dentro do ambiente da turbina.

O desafio deste projeto está em conseguir instalar o robô com segurança e
praticidade, no ambiente confinado e de difícil acesso da unidade geradora. Por
conta do único acesso ser através de uma escotilha de $850~mm$ de diâmetro, o
robô deve ser de médio a pequeno porte, ter uma base para sua fixação e
posicionamento próximo a pá, que seja leve e modular, para fácil transporte,
montagem e acesso pela escotilha. O desafio torna-se ainda maior se considerado
que o processo de revestimento tem requisitos que devem ser controlados com
precisão, para garantir sua qualidade. Este tipo de revestimento só pode ser
aplicado roboticamente porque desvios de posição, orientação ou velocidade do
processo, podem simplesmente reprovar ou inviabilizar o procedimento.

Por isso, considerando todas as restrições do sistema, projetou-se para a
solução \textit{in situ}, uma base metálica, leve e modular, que possibilitasse
o posicionamento do robô próximo às pás, em diversas configurações de montagem.
Porém, simulações estáticas de carregamento, realizadas por Anáise por Elementos
Finitos apresentaram resultados que o comportamento dinâmico da estrutura,
poderia causar grandes amplitudes de deslocamento.
Neste momento percebeu-se a necessidade de ter um modelo capaz de simular a
tarefa de revestimento e obter-se assim o comportamento dinâmico acoplado entre
base e robô, para então quantificar os desvios causados aos parâmetros
controlados do processo de revestimento.

A busca por linhas de pesquisa sobre robôs em bases flexíveis revelou que o tema
foi explorado principalmente sob um ponto de vista de manipuladores macro-micro,
sistemas em que consideram a flexibilidade do manipulador, e de métodos de
controle ativos para amortecer as vibrações. Nestes estudos, carece uma
metodologia mais sistemática para avaliação dos efeitos da rigidez e
amortecimento da base, e sua relação com a geometria e materiais da estrutura.
Assim como, dos erros em relação aos parâmetros controlados de determinado
processo, causados pela dinâmica da base. Esta pesquisa tem como motivação aliar
o método de avaliação do sistema robô-base a aplicações práticas da indústria,
tendo como principal estudo de caso o sistema de revestimento robótico do
projeto EMMA.

De modo geral, pode-se afirmar que a ação do robô ao realizar
determinada tarefa irá originar, por reação, deslocamentos dinâmicos em sua
base. Estes deslocamentos causarão algum impacto na tarefa a ser realizada como:
mudanças de orientação, velocidade e posicionamento da ferramenta acoplada ao
manipulador. Estes efeitos não são considerados nos métodos de controle
convencionais das juntas do manipulador, porque não podem ser previstos sem que
haja alguma instrumentação periférica, capaz de medir e enviar tais informações
ao controlador.

Caso a base não ofereça a rigidez necessária, a amplitude dos deslocamentos pode
simplesmente inviabilizar a tarefa desejada e, se possível, modificações na
estrutura da base ou uma nova estrutura devem ser projetadas. Por isso, ser
capaz de estimar o comportamento do sistema, por simulação no computador, é uma
importante ferramenta, que pode principalmente evitar o projeto e fabricação de
uma solução equivocada.


% O amortecimento da estrutura que serve de base para o manipulador tem um papel
% importante na dinâmica do sistema. Como foi demonstrado, o modelo MBS
% acoplado robô-base requer como entrada os parâmetros de inércia, rigidez e amortecimento
% da base. Os dados de inércia são facilmente obtidos pelo modelo CAD e a matriz
% de inércia pela Análise por Elementos Finitos. Entretanto, estas
% ferramentas não são capazes de fornecer informações sobre o amortecimento.
% Modelar o amortecimento é um problema constante na engenharia de sistemas
% mecânicos e continua um desafio.
% 
% O sistema robô-base não possui nenhum elemento com a finalidade específica de
% controle ou absorção de vibrações. O que ocorre é o citado amortecimento
% estrutural (seção~\ref{sec::amortecimento}), formado principalmente pelo
% amortecimento do material \todo{Inlcuir referência no texto} e por atrito,
% devido a folgas nos elementos de conexão, parafusos por exemplo.
% 
% Apesar de existir métodos para estimar o amortecimento material e de atrito
% analiticamente, estes são adequados para estruturas muito simples e uniformes,
% como vigas, placas ou corpos de prova. A estrutura que forma a base do robô é
% relativamente muito complexa, possuindo diferentes materiais, conexões e formas,
% o que torna impossível modelar analiticamente o amortecimento.
% 
% Logo, recorre-se ao método experimental para obter o amortecimento da base.
% A vantagem é que por este método, obtém-se os parâmetros do comportamento total
% da estrutura, não importando as infinitas interações internas dos materiais e de
% atrito impossíveis de se medir ou quantificar individualmente.
% Mais especificamente, o que importará mesmo é o comportamento do ponto de
% interação robô-base.

% O objetivo do projeto desta plataforma não era o de oferecer grande
% flexibilidade, mas pelo contrário, ser suficientemente rígida para operar o robô
% com segurança durante testes do projeto EMMA. A análise deste modelo de base
% será interessante para comparar com o modelo rígido e verificar se o projeto,
% que foi realizado antes de existir o método proposto neste trabalho,
% atende a este requisito.


% -.~.-.~.-.~.-.~.-.~.-.~.-.~.-.~.-.~.-.~.-.~.-
\section{Objetivos}

Este trabalho tem como objetivo oferecer um método para calcular e quantificar
os deslocamentos teóricos da base, quando excitada pelo movimento do manipulador
robótico, e assim calcular os erros de posicionamento, velocidade e orientação
no efetuador do robô, ao longo de uma trajetória. 

Também oferece uma forma de avaliar o efeito, no resultado da trajetória, de
alterações estruturais na base. Se resultarem erros fora de uma margem de
tolerância admitida, o método pode ser utlizado iterativamente, realizando-se
modificações estratégicas da base -- como materiais, configuração geométrica,
apoios, reforço estrutural, massa, etc.
O objetivo deste processo é, em um ambiente de simulação, verificar a nova base,
e a partir dos resultados fazer as modificações até atingir a tolerância que
satisfaz a qualidade da tarefa.

Por fim, tem-se por objetivo discutir estratégias futuras de instrumentação e
controle, para levar em consideração as perturbações causadas pela base e
corrigir as diferenças de posicionamento, velocidade e orientação das juntas do
manipulador.


% -.~.-.~.-.~.-.~.-.~.-.~.-.~.-.~.-.~.-.~.-.~.-
\section{Ferramentas utilizadas}

De maneira geral, o repertório de programas utilizados compõe um conjunto de
ferramentas importantes para o método, como: Álgebra Computacional Simbólica
(CAS), Análise por Elementos Finitos (AEF), Desenho Assistido por Computador
(CAD), aquisição e tratamento de dados experimentais. Esta pesquisa também tem
como preocupação permitir ao usuário utilizar as ferramentas que estiverem à sua
disposição pessoal ou da instituição (laboratório, empresa, universidade, etc.),
não limitando ao uso de qualquer \textit{software} em especial. Embora tenham
sido utilizados programas de alto nível ficará claro que outros também poderiam
ter sido utilizados. Por isso, são listados a seguir os programas escolhidos e
também alternativas disponíveis no mercado.

\subsubsection{Sophia-Maple}

Sophia é um conjunto de rotinas, desenvolvido por \citet{lesser1995analysis}
para modelar sistemas mecânicos multicorpos e derivar suas equações de
movimento. Foi criado originalmente para Maple, depois foi adaptada uma uma
versão para o Mathematica.
Maple\cite{maple} é um programa de álgebra simbólica computacional (CAS), com
ênfase em manipulações algébricas simbólicas ao invés de numéricas, o que
permite modelar e solucionar sistemas dinâmicos simbolicamente. O pacote Sophia,
utilizado com o Maple será descrito apenas por Sophia-Maple.

Alternativas:
%
\begin{enumerate*}
	\item Mathematica;
	\item SymPy;
\end{enumerate*}

\subsubsection{SolidWorks}

SolidWorks\cite{solidworks} é um programa de CAD 3D, utilizado principalmente
para modelagem de peças e conjuntos mecânicos.
Este programa tem interface amigável que facilita o desenho de conjuntos e
montagens complexas. Permite também a definição do material de cada peça, por
fornecer vasta lista de materiais e suas propriedades físicas e mecânicas.
Realiza automaticamente o cálculo, por exemplo, de massa, centro de
gravidade, momentos de inércia e propriedades de seção transversal, que são
muito úteis para o método. Também permite a conversão do modelo 3D para diversos
formatos CAD, que facilitam a exportação de versões para Análise por Elementos
Finitos para outros programas.

Alternativas:
%
\begin{enumerate*}
	\item AutoCAD;
	\item CATIA;
	\item Pro/ENGINEER.
\end{enumerate*}

\subsubsection{Simulation Mechanical}

O Simulation Mechanical\cite{autodesk} é um programa para Análise por Elementos
Finitos distribuido pela Autodesk.  Realiza análises estáticas lineares e não-lineares,
análise modal, análise de carregamento transiente, análises térmicas e de
mecânica dos fluidos.
É utilizado nesta pesquisa para simulações de carregamento estático linear, a
fim de determinar a matriz de rigidez da base do manipulador. O programa permite
grande variedade de formatos para importação de modelos CAD, o que facilita a
integração com o SolidWorks.

Alternativas:
%
\begin{enumerate*}
	\item Abaqus;
	\item Ansys;
	\item SolidWorks Simulation.
\end{enumerate*}


\subsubsection{SignalExpress e ME`Scope VES}

Para determinação de parâmetros modais da base do robô é necessário obter dados
experimentais através de instrumentação da base. São utilizados 2 programas: o
primeiro para aquisição, visualização e tratamento dos dados; o segundo para
cálculo dos parâmetros modais.

SignalExpress\cite{signalexpress} é programa aquisição, análise e
apresentação de dados compatível com diversos instrumentos e sistemas de
aquisição e medição experimentais e operacionais. 
É utilizado nesta pesquisa para adquirir e processar os dados do ensaio
experimental de uma versão de testes da base do robô.

O ME'Scope VES (\textit{Visual Engineering Series})\cite{mescope} é um programa
para facilitar a visualização e análise de vibrações de máquinas e estruturas
mecânicas, utilizando os dados reais registrados experimentalmente. O programa
também realiza análise modal, a partir das Funções de Resposta em Frequência
(FRF's) dos dados importados. É utilizado portanto para  determinar o
amortecimento modal, parâmetro necessário ao modelo teórico da base.

Alternativas:
%
\begin{enumerate*}
	\item Brüel \& Kjær PULSE;
	\item OROS Modal Analysis;
	\item LabVIEW.
\end{enumerate*}

\section{Notação}

% Texto introdutório considerando a pluralidade de referências.
