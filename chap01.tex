\chapter{Introdução}

% Segundo a norma de formata{\c c}\~ao de teses e disserta{\c c}\~oes do
% Instituto Alberto Luiz Coimbra de P\'os-gradua{\c c}\~ao e Pesquisa de
% Engenharia (COPPE), toda abreviatura deve ser definida antes de
% utilizada.\abbrev{COPPE}{Instituto Alberto Luiz Coimbra de P\'os-gradua{\c
% c}\~ao e Pesquisa de Engenharia}
% 
% Do mesmo modo, \'e imprescind\'ivel definir os s\'imbolos, tal como o
% conjunto dos n\'umeros reais $\mathbb{R}$ e o conjunto vazio $\emptyset$.
% \symbl{$\mathbb{R}$}{Conjunto dos n\'umeros reais}
% \symbl{$\emptyset$}{Conjunto vazio}

% -.~.-.~.-.~.-.~.-.~.-.~.-.~.-.~.-.~.-.~.-.~.-
\section{Motivação}

A ação dinâmica do robô ao realizar determinada tarefa irá originar, por reação,
deslocamentos dinâmicos em sua base. Estes deslocamentos causarão algum impacto
na tarefa a ser realizada como mudanças de orientação, velocidade e
posicionamento da ferramenta acoplada ao manipulador. Estes efeitos não são
considerados nos métodos de controle convencionais das juntas do manipulador,
porque não podem ser previstos sem que haja alguma instrumentação periférica,
capaz de medir e enviar as informações ao controlador.

O método proposto tem como preocupação oferecer uma metodologia que permite ao
usuário utilizar as ferramentas que estão à sua disposição pessoal ou da
instituição (laboratório, empresa, universidade, etc.), não limitando ao uso de
qualquer \textit{software} específico. Apesar de terem sido utlilizados
programas de alto nível como Maple, Autodesk Simulation e SolidWorks, ficará
claro que diversos \textit{softwares} de Álgebra Computacional Simbólica (CAS),
Análise de Elementos Finitos (FEA) e Desenho Assistido por Computador (CAD),
respectivamente, poderiam ter sido utilizados.

O amortecimento da estrutura que serve de base para o manipulador tem um papel
importante na dinâmica do sistema. Como foi demonstrado, o modelo MBS
acoplado robô-base requer como entrada os parâmetros de inércia, rigidez e amortecimento
da base. Os dados de inércia são facilmente obtidos pelo modelo CAD e a matriz
de inércia pela Análise por Elementos Finitos. Entretanto, estas
ferramentas não são capazes de fornecer informações sobre o amortecimento.
Modelar o amortecimento é um problema constante na engenharia de sistemas
mecânicos e continua um desafio.

O sistema robô-base não possui nenhum elemento com a finalidade específica de
controle ou absorção de vibrações. O que ocorre é o citado amortecimento
estrutural (seção~\ref{sec::amortecimento}), formado principalmente pelo
amortecimento do material \todo{Inlcuir referência no texto} e por atrito,
devido a folgas nos elementos de conexão, parafusos por exemplo.

Apesar de existir métodos para estimar o amortecimento material e de atrito
analiticamente, estes são adequados para estruturas muito simples e uniformes,
como vigas, placas ou corpos de prova. A estrutura que forma a base do robô é
relativamente muito complexa, possuindo diferentes materiais, conexões e formas,
o que torna impossível modelar analiticamente o amortecimento.

Logo, recorre-se ao método experimental para obter o amortecimento da base.
A vantagem é que por este método, obtém-se os parâmetros do comportamento total
da estrutura, não importando as infinitas interações internas dos materiais e de
atrito impossíveis de se medir ou quantificar individualmente.
Mais especificamente, o que importará mesmo é o comportamento do ponto de
interação robô-base.

\todo[inline]{Falar do projeto EMMA e da base de testes}

O objetivo do projeto desta plataforma não era o de oferecer grande
flexibilidade, mas pelo contrário, ser suficientemente rígida para operar o robô
com segurança durante testes do projeto EMMA. A análise deste modelo de base
será interessante para comparar com o modelo rígido e verificar se o projeto,
que foi realizado antes de existir o método proposto neste trabalho,
atende a este requisito.


% -.~.-.~.-.~.-.~.-.~.-.~.-.~.-.~.-.~.-.~.-.~.-
\section{Objetivos}

Este trabalho tem como objetivo oferecer um método para calcular e quantificar
os deslocamentos teóricos da base, e assim calcular os erros de posicionamento,
velocidade e orientação no seu efetuador. Isto é feito a partir da
quebra do paradigma da base como uma estrutura idealmente rígida -- ou seja, de
rigidez tão grande que pudessem ser desconsiderados quaisquer efeitos da ação do
manipulador -- e considerando-a como um corpo elástico, com flexibilidade e
amortecimento.

Quantificando-se o comportamento dinâmico do sistema base e robô, pode-se então
avaliar os desvios causados na trajetória. Se o resultado demonstrar erros fora
de uma margem de tolerância admitida, o método pode ser utlizado iterativamente,
alterando-se parâmetros da base.
Após mudanças estratégicas destes parâmetros -- como materiais, configuração
geométrica, apoios, contravetamentos, massa, etc. -- obtém-se novos resultados.
O objetivo deste processo é, em um ambiente de simulação, verificar a nova base,
e a partir dos resultados fazer as modificações até atingir a tolerância que
satisfaz a qualidade da tarefa.

Além do método iterativo, tem-se por objetivo discutir estratégias de
instrumentação e controle, de forma a aperfeiçoar o método original, para levar
em consideração as perturbações causadas pela base e corrigir as diferenças de
posicionamento, velocidade e orientação das juntas do manipulador.


% -.~.-.~.-.~.-.~.-.~.-.~.-.~.-.~.-.~.-.~.-.~.-
\section{Ferramentas utilizadas}

% De maneira geral, o repertório de programas utilizados atende a objetivos
% específicos previstos nesta pesquisa. Em geral, no entanto, se inserem nas
% frentes de simulação, previsão aquisição e pós-processamento.


\subsubsection{Sophia-Maple}

\subsubsection{SolidWorks}

\subsubsection{Autodesk Simulation}

\subsubsection{SignalExpress}

\subsubsection{ME`Scope}


\section{Notação}

% Texto introdutório considerando a pluralidade de referências.
