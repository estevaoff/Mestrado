\chapter{Método Proposto}



% -.~.-.~.-.~.-.~.-.~.-.~.-.~.-.~.-.~.-.~.-.~.-
\section{Visão geral}

O método proposto pode ser dividido em 5 etapas:
%
\begin{enumerate}
  \item Modelo dinâmico do robô (MBS Robô);
  \item Modelo dinâmico da base (MBS Base);
  \item Ensaio Experimental da Base;
  \item Modelo dinâmico acoplado (MBS Robô-Base);
  \item Cálculo dos erros teóricos.
\end{enumerate}

Os modelos dinâmicos -- robô, base e  acoplado -- são apresentados nas
seções~\ref{sec::robo},~\ref{sec::base} e ~\ref{sec::acoplado}, onde
demonstra-se o desenvolvimento da cinemática e dinâmica destes sistemas
multicorpos.
Com o auxílio do \textit{software} de álgebra computacional Maple, utilizando as
rotinas do Sophia, revela-se uma forma prática e sistemática de modelar o
sistema multicorpos de qualquer número de graus de liberdade e, pelo método de
Kane, deduzir-se rapidamente as equações de movimento dos sistemas.

O MBS Robô configura o manipulador como sobre uma base ideal, ou perfeitamente
rígida, já que em sua origem não se oferece nenhum grau de liberdade. Com este
modelo realiza-se, portanto, as simulações dinâmicas das trajetórias ``ideais'',
ou seja, que seriam obtidas numa base perfeitamente rígida. Este resultado
fornecere as posições, velocidades e orientações que serão referência para
comparação destes mesmos parâmetros resultantes do modelo acoplado.

Para modelagem da base, é necessário fornecer as matrizes que representam sua
rigidez e amortecimento. Na seção~\ref{sec::base} é demonstrado um método de se
obter a matriz de rigidez por meio de Análise por Elementos Finitos (AEF).

Na seção~\ref{sec::experimento} apresenta-se como obter os parâmetros modais da
base por meio de ensaio de vibrações. Os dados são tratados para obter as
Funções de Resposta em Frequência (FRF's) e então estimar seus parâmetros
modais. O resultado fornece a matriz de amortecimento da base de teste que será
utilizada no modelo MBS acoplado.

O MBS Robô-Base considera os 6 graus de liberdade da base (3 tranlações e 3
rotações) e os 5 graus de liberdade do robô, formando um sistema único acoplado
de 11 gdl. É simulada a mesma trajetória fornecida no modelo rígido, com o
objetivo de comparar os resultados quando o robô opera sobre uma base rígida em
relação a mesma operação do robô sobre base flexível. São então verificados os
efeitos da elasticidade da base sobre os parâmetros controlados do processo
sendo executado.

A Figura~\ref{fig::visgeral} resume a visão geral
do método proposto facilitando a visualização das relações entre os modelos.

\begin{figure}[h]
	\centering 
 	\includegraphics[width=0.99\textwidth]{figs/visgeral}
 	\caption{Diagrama da visão geral do método}
 	\label{fig::visgeral}
\end{figure}



% -.~.-.~.-.~.-.~.-.~.-.~.-.~.-.~.-.~.-.~.-.~.-
\section{Modelo do robô} \label{sec::robo}

% Nesta seção, são detalhados os procedimentos para representar o manipulador
% robótico como um conjunto MBS e utilizá-lo para simular as trajetórias
% referentes a uma determinada tarefa. O manipulador será descrito pelo conjunto
% de Sistemas de Coordenadas (SC's) referente a cada uma de suas juntas, pelas
% distâncias entre os SC's e posição dos centros de massa de cada elo, e pelos
% parâmetros de massa e momento de inércia de cada elo. Os elos do robô e a
% ferramenta acoplada no efetuador representam cada corpo do sistema MBS. A
% modelagem do manipulador é simplificada utilizando as rotinas de CAS
% desenvolvida especialmente para MBS, o Sophia, assim como a notação algébrica de
% Lesser, apresentada na seção~\ref{sec::sophia-kane}.

\subsection{Descrição do braço robótico} \label{sec::descricao_mh12}

O manipulador escolhido para estudo é o mesmo que será utilizado no projeto EMMA
para revestimento de superfícies metálicas por HVOF. Trata-se de um robô
comercial da série MOTOMAN, modelo MH12, fabricado pela Yaskawa Motoman
apresentado na Figura~\ref{fig::mh12_foto}.

\begin{figure}[h]
    \centering
    \begin{subfigure}[b]{0.3\textwidth}
        \includegraphics[width=\textwidth]{figs/mh12_foto}
        \caption{MOTOMAN MH12. \\Fonte: adaptada de~\cite{manualmh12}}
        \label{fig::mh12_foto}
    \end{subfigure}
    \quad %add desired spacing between images, e. g. ~, \quad, \qquad, \hfill
    % etc.
      %(or a blank line to force the subfigure onto a new line)
    \begin{subfigure}[b]{0.5\textwidth}
        \includegraphics[width=\textwidth]{figs/mh12_diagram}
        \caption{Diagrama dos elos e juntas. \\Fonte: adaptada de~\cite{manualmh12}}
        \label{fig::mh12_diagram}
    \end{subfigure}
    \caption{Manipulador robótico para modelo}\label{fig::resumo_mh12}
\end{figure}

\begin{table}[h]
\centering
\caption{Sistemas de coordenadas, elos e coordenadas generalizadas}
\label{tab::resumo_mh12}
\begin{tabular}{@{}clc@{}}
\toprule
SC & Elo              & \multicolumn{1}{l}{Coord. gen. associada} \\ \midrule
Z  & Pedestal do robô & -                                         \\
S  & \textit{Swivel}  & q1                                        \\
L  & Braço inferior   & q2                                        \\
U  & Braço superior   & q3                                        \\
R  & Braço de rolagem & q4                                        \\
B  & Pulso            & q5                                        \\
T  & Efetuador        & q6                                        \\ \bottomrule
\end{tabular}
\end{table}

Este robô é do tipo braço antropomórfico de 6 juntas rotacionais e portanto 6
graus de liberdade (6 gdl), contendo o último elo um porta-ferramentas que
suporta uma carga útil de até 12 kg. A Figura~\ref{fig::mh12_diagram} apresenta
os nomes dos elos e coordenadas generalizadas associados a cada um dos sistemas
de coordenadas e são resumidos na Tabela~\ref{tab::resumo_mh12}.

O alcance horizontal deste manipulador chega a 1,440~m, e vertical a
2,511~m. Estão representados no diagrama do espaço de trabalho na
Figura~\ref{fig::workspace} em que a área sombreada é formada por todos os
pontos alcançáveis pelo manipulador, dentro dos limites de cada junta.

\begin{figure}[h]
	\centering 
 	\includegraphics[width=0.7\textwidth]{figs/workspace}
 	\caption[Vistas lateral e superior do espaço de trabalho]{Vistas lateral e
 	superior do espaço de trabalho \\Fonte: adaptada de~\cite{manualmh12}}
 	\label{fig::workspace}
\end{figure}

Conforme discutido na seção~\ref{sec::manind}, este tipo de braço robótico
permite desacoplar o sistema em 2 sub-problemas: posicionamento e orientação.
Logo, para simplificar o modelo e o cálculo da cinemática inversa, serão consideradas
as 3 primeiras juntas para posicionamento e 3 últimas (pulso esférico) para
orientação da ferramenta.

A última junta, no efetuador, tem a finalidade de orientar a ferramenta em torno
do seu eixo axial. Como o processo de revestimento por HVOF independe desta
orientação, esta junta \emph{não será incluída}, mantendo este acoplamento
rígido, o que transforma os dois últimos elos em apenas um corpo.
Como resultado, tem-se um sistema de 5 gdl.


\subsection{Cinemática Direta}\label{sec::dkin}

Como foi discutido na seção~\ref{sec::cinematica}, o procedimento mais utilizado
para a modelagem cinemática de manipuladores robóticos é o método dos parâmetros
de Denavit-Hartenberg. Apesar de sua popularidade e vasta utilização na
modelagem cinemética de manipuladores variados, o pacote Sophia-Maple será
utilizado, uma vez que já fornece praticidade e flexibilidade para derivar as
equações cinemáticas de sistemas multicorpos, sem nenhum aumento de
complexidade. Pode-se dizer inclusive que este método é vantajoso em relação ao
DH, por não restringir a escolha dos referenciais, permitindo investigar
configurações que melhorem a eficiência das manipulações algébricas.

Demonstra-se portanto os procedimentos para modelagem de um manipulador robótico
pelas rotinas do Sophia-Maple. A cada etapa são apresentadas as linhas de código
correspondentes, a fim de ilustrar a praticidade do método.

\subsubsection{Sistemas de Coordenadas}

A primeira etapa para obter-se as equações cinemáticas será definir o sistema de
coordenadas fixo em cada elo do robô. A Figura~\ref{fig::scs} é um modelo CAD do
manipulador e apresenta a posição de cada SC, na sua configuração inicial.

\begin{figure}[h]
    \centering
    \begin{subfigure}[b]{0.20\textwidth}
        \includegraphics[width=\textwidth]{figs/sc_front}
        \caption{Vista frontal}
        \label{fig::sc_front}
    \end{subfigure}
    \quad %add desired spacing between images, e. g. ~, \quad, \qquad, \hfill
    % etc.
      %(or a blank line to force the subfigure onto a new line)
    \begin{subfigure}[b]{0.7\textwidth}
        \includegraphics[width=\textwidth]{figs/sc_lat}
        \caption{Vista lateral}
        \label{fig::sc_lat}
    \end{subfigure}
    \caption{Sistemas de coordenadas do robô}\label{fig::scs}
\end{figure}

Na vista frontal (Figura~\ref{fig::sc_front}) nota-se que foi escolhida uma
configuração em que todos os SC's estivessem no mesmo plano xz. Outra observação
é que os SC's R, B e T estão fixados no mesmo ponto. que representa a origem do
``pulso'' do braço robótico. Estas considerações reduzem a quantidade de termos
das equações cinemáticas. Também terão grande impacto no cálculo da cinemática
inversa, como será visto na seção~\ref{sec::ikin_mh12}.

Logo, pode-se escrever as relações entre estes referenciais em termos das
coordenadas generalizadas do sistema, para obter as transformações entre cada
SC. No Sophia-Maple isto é feito com o uso da função \texttt{chainSimpRot}, da
seguinte forma:

\bigskip \noindent {\tt > chainSimpRot( [Z,S,1,q1], [S,L,3,q2], [L,U,3,q3],
[U,R,2,q4], [R,B,3,q5], [B,T,2,q6] )} \bigskip 

Esta função cria as matrizes de rotação entre os referenciais do sistema,
informando a cada transformação, o eixo de rotação
(onde 1=X, 2=Y e 3=Z) e a coordenada generalizada associada
(q1,~\ldots~,q6). Foi utilizada a forma de coordenadas relativas entre cada SC.

Como exemplo, considere o termo {\tt [L,U,3,q3]}. Representa uma rotação de um
ângulo q3, do SC-U em relação ao SC-L, em torno do eixo Z.
Logo, a matriz Transformação Homogênea entre o referencial inercial Z e o braço
superior U seria:
%
$$ R_{Z}^{U} = R_{Z}^{S} R_{S}^{L} R_{L}^{U} $$
%
A função \texttt{Rmx} do Sophia-Maple retorna a matriz de rotação entre quaisquer
sistemas de coordenadas. No Sophia-Maple, escreve-se:

\bigskip \noindent {\tt > Rmx(Z,U)} \bigskip 

E obtém-se o resultado:
%
$$ R_{Z}^{U} = \left( \begin {array}{ccc} {\it c2}\,{\it c3}-{\it s2}\,{\it
s3}&-{ \it c2}\,{\it s3}-{\it s2}\,{\it c3}&0\\ \noalign{\medskip}{\it c1}\,{
\it c2}\,{\it s3}+{\it c1}\,{\it s2}\,{\it c3}&{\it c1}\,{\it c2}\,{
\it c3}-{\it c1}\,{\it s2}\,{\it s3}&-{\it s1}\\ \noalign{\medskip}{
\it s1}\,{\it c2}\,{\it s3}+{\it s1}\,{\it s2}\,{\it c3}&{\it s1}\,{
\it c2}\,{\it c3}-{\it s1}\,{\it s2}\,{\it s3}&{\it c1}\end {array}
 \right) $$
 %
 A matriz acima foi escrita em notação trigonométrica simplificada, em que c1,
 s1, ,c2, s2, \ldots, e assim por diante, representam as funções senos e
 cossenos dos ângulos das coordenadas generalizadas q1, \ldots, q5. Para melhor
 visualização das matrizes e equações, esta notação será utilizada em todo o
 texto.
 
\subsubsection{Vetores posição e centros de massa}

A segunda etapa é definir os vetores posição dos centros de massa de cada corpo.
Para isso, define-se auxiliarmente os vetores posição entre cada sistema de
coordenada, desde o referencial inercial SC-Z até o efetuador em SC-T, de
acordo com a equação~\ref{eq::posjnutas}.

Logo, de maneira geral, pode-se escrever a posição de qualquer ponto pela
seguinte relação:
%
\begin{align}
	^{Z}\mathbf{p}^{k} &= ~^{Z}\mathbf{p}^{k-1} + ~^{k-1}\mathbf{p}^{k}
	\label{eq::posjnutas} \\
	^{Z}\mathbf{pcm}^{k} &= ~^{Z}\mathbf{pcm}^{k-1} + ~^{k}\mathbf{pcm}^k
	\label{eq::pcm}
\end{align}
%
Onde $^{Z}\mathbf{p}^{k}$ é o vetor posição do referencial $k$ em relação ao
referencial inercial $Z$ e analogamente $^{Z}\mathbf{pcm}^{k}$ é o vetor
posição do centro de massa do corpo $k$, tal que $k$ varia em \{S,L,U,B,T\},
sistemas de coordenadas do robô.
No Sophia-Maple é utilizada a notação de Lesser, por meio dos \texttt{Evectors},
para representar esses vetores, como descrito na seção~\ref{sec::sophia-kane}.

\bigskip \noindent {\tt > pZ:= Evector(0,0,0,Z)} \\
\noindent {\tt > for k from corpo[1] to corpo[6] do \\
\indent p||k:= p||{k-1} \&++ Evector(p||{k-1}||{k}||x, p||{k-1}||{k}||y,
\indent p||{k-1}||{k}||z, k-1) \\
end do:} \bigskip 

E para os centros de massa:

\bigskip \noindent {\tt > for k from corpo[1] to corpo[6] do \\
\indent pcm||k:= p||{k} \&++ Evector(pcm||{k}||x, pcm||{k}||y, pcm||{k}||z, k)
\\ end do:} \bigskip 

A Figura~\ref{fig::amb3D} apresenta o ambiente de simulação 3D, que é utilizado
para facilitar a visualização do posicionamento e movimento das juntas e elos do
robô. Definidos os pontos de origem de cada referencial, representa-se os elos
do manipulador por uma linha colorida entre os SC's. A ferramenta acoplada é
representada por uma linha que se estende desde a origem do pulso até sua
extremidade.

Neste ambiente 3D é possível fornecer qualquer configuração de juntas e
obter-se um resultado visual de posicionamento e orientação dos elos. A
Figura~\ref{fig::amb3D_posA} ilustra o resultado para a configuração inicial do
do robô, $q1 = \ldots = q6 = 0$; e a Figura~\ref{fig::amb3D_posB} para a
configuração $q1 = \ldots = q6 = {-}^{\pi}/_2$.

\begin{figure}[h]
    \centering
    \begin{subfigure}[b]{0.4\textwidth}
        \includegraphics[width=\textwidth]{figs/amb3D_posA}
        \caption{Configuração inicial}
        \label{fig::amb3D_posA}
    \end{subfigure}
    \quad %add desired spacing between images, e. g. ~, \quad, \qquad, \hfill
    % etc.
      %(or a blank line to force the subfigure onto a new line)
    \begin{subfigure}[b]{0.45\textwidth}
        \includegraphics[width=\textwidth]{figs/amb3D_posB}
        \caption{Configuração $q1 = \ldots = q6 = {-}^{\pi}/_2$}
        \label{fig::amb3D_posB}
    \end{subfigure}
    \caption{Ambiente 3D de visualização do robô}\label{fig::amb3D}
\end{figure}

\subsubsection{Velocidades}

As velocidades dos centros de massa de cada corpo são calculadas como a derivada
do vetor posição com respeito ao referencial inercial. Logo, a derivada da
equação~\ref{eq::pcm} em Z:
%
\begin{equation} \label{eq::veloc}
	\mathbf{v}^{k} = \frac{^{Z}\mathrm{d} }{\mathrm{d} t}
	\mathbf{pcm}^{k} 
\end{equation}
%

O Sophia-Maple possui um conjunto de rotinas para diferenciação dos
\texttt{Evectors} com relação ao referencial inercial, por meio da função
\texttt{cdft}, que reconhece automaticamente o referencial em que cada vetor
está descrito e realiza o cálculo diferencial vetorial. Logo, pode-se escrever o
seguinte comando para definir as velocidades de acordo com a
equação~\ref{eq::veloc}:

\bigskip \noindent {\tt > for k from corpo[1] to corpo[6] do \\
\indent v||k:= cdft(pcm||{k}, Z) \\
end do:} \bigskip 

As velocidades angulares serão a taxa de variação angular das matrizes de
rotação entre os referenciais. Serão definidas as velocidades angulares de cada
corpo em relação ao referencial inercial. Conforme apresentado na
seção~\ref{sec::cinematica} pode-se escrever a velocidade angular de uma matriz
rotação $R(t)$ como:
%
\begin{equation} \label{eq::velang}
	\frac{\mathrm{d} R}{\mathrm{d} t} = S \cdot R(t)
\end{equation}
%
Onde $S$ é o tensor velocidade angular associado a $\omega$.
%
\begin{align}
% eq1
& S = \begin{pmatrix}
0				& -\omega_{z}(t)	& \omega_{y}(t) \\ 
\omega_{z}(t)	& 0					& -\omega_{x}(t) \\ 
-\omega_{y}(t)	& \omega_{x}(t)		& 0
\end{pmatrix} \\
% eq2
& \boldsymbol{\omega} = [\omega_{x}, \omega_{y}, \omega_{z}]^{T}
\end{align}
%
No Sophia-Maple o tensor $S$ pode ser calculado através da função
\&\texttt{VtoD}. Por exemplo, o tensor entre os referenciais SC-L e SC-S seria:
%
$$
S_{L}^{Z} = \left( \begin {array}{ccc} 0&-{\it q2t}&-\sin \left( {\it q2}
 \right) {\it q1t}\\ \noalign{\medskip}{\it q2t}&0&-\cos \left( {\it 
q2} \right) {\it q1t}\\ \noalign{\medskip}\sin \left( {\it q2}
 \right) {\it q1t}&\cos \left( {\it q2} \right) {\it q1t}&0
\end {array} \right)
$$
%
Porém, pode-se ainda fazer o uso da função \texttt{aV} que
calcula diretamente o vetor velocidade angular $^{A}\omega^{B}$ entre dois
referenciais. O que retornaria o seguinte vetor:
%
$$
^{Z}\boldsymbol{\omega}^{L} = [-{\it q1t},\sin \left( {\it q1} \right) {\it q2t},-\cos
\left( {\it q1} \right) {\it q2t}]
$$
%
No caso do manipulador, são calculadas as velocidades de cada corpo em relação
ao referencial Z. São definidos da seguinte forma:

\bigskip \noindent {\tt > for k from corpo[1] to corpo[6] do \\
\indent w||k:= aV(Z, corpo[k]) \\
end do:} \bigskip 

Podemos finalmente escrever o vetor Velocidades Generalizadas, que nada mais é
que um \texttt{Kvector} que fornece a lista de \texttt{Evectors} referente às
velocidades lineares e angulares calculadas. Logo, o vetor velocidades
generalizadas tem a seguinte forma:
%
\begin{equation}
	\mathbf{v}_{g} = [ \mathbf{v}^{S},\ldots, \mathbf{v}^{T}, \boldsymbol{\omega}^{S},\ldots,
	\boldsymbol{\omega}^{T}, 12]
\end{equation}
%
Onde cada termo do \texttt{Kvector} é um \texttt{Evector} da velocidade
generalizada.
Por fim, o último termo representa a quantidade de \texttt{Evectors} que formam o
\texttt{Kvector}.

\subsubsection{Hiperplano tangente}

Uma vez que as velocidades generalizadas foram definidas pode-se obter o
conjunto de vetores tangentes, pelo cálculo das velocidades parciais, que formam
o chamado hiperplano tangente.

A função \texttt{KMtangents} do Sophia-Maple obtém os termos que formam os
vetores tangentes em \texttt{Kvectors} para cada coordenada generalizada. O
conjunto que forma o hiperplano tangente introduz uma estrutura do Sophia, dos
super \texttt{Kvectors}, ou \texttt{SKVector}. Esta estrutura facilitará a
projeção das equações dinâmicas no hiperplano tangente. Logo, para o manipulador
escreve-se:

\bigskip \noindent {\tt > tau:= KMtangents(vK,u,6):} \bigskip 


\subsection{Dinâmica}

Definida a cinemática direta do manipulador, deseja-se obter as forças de
inércia e externas generalizadas que fornecerão as equações de movimento do
sistema. Nesta seção primeiramente são introduzidos os parâmetros de inércia de
cada corpo do manipulador e como foram obtidos.

\subsubsection{Parâmetros de inércia do manipulador robótico}

Os parâmetros de inércia são a massa e o momento de inércia de cada corpo do
sistema. É importante notar que não é possível obter um modelo dinâmico preciso
apenas com as informações contidas nos manuais e fichas técnicas do manipulador
robótico. Infelizmente os parâmetros de inércia individuais de cada corpo não
são fornecidos pelos fabricantes e, portanto, serão estimados pelo método a
seguir.

Do manual do MOTOMAN MH12~\cite{manualmh12}, informa-se a massa total do robô,
de 130~kg. Também é disponibilizado, diretamente pelo fabricante, um modelo CAD
com ótimo detalhamento de cada corpo individualmente.

O programa SolidWorks foi utilizado para obter informações importantes para a
estimativa dos parâmetros de inércia. Nele é possível obter o volume de cada elo
em separado. Com isso, pode-se calcular uma massa específica média do robô, de
acordo com a equação~\ref{eq::rho}.
\begin{align}
	\rho_{m} = M_{total}\sum_{corpo[i]}^{corpo[n]} V_{i} \label{eq::rho} \qquad & ,
	i = Z,S,L,U,R,B,T
\end{align}
%
Em seguida, estima-se a massa de cada corpo, calculada pelo produto da massa
específica média $\rho_{m}$ e o volume, de acordo com a
equação~\ref{eq::massai}.
%
\begin{align}
	m_{i} = \rho_{m} \cdot V_{i} \label{eq::massai} \qquad &, i =
	Z,S,L,U,R,B,T
\end{align}
%
A massa da ferramenta acoplada ao efetuador foi obtida de forma precisa,
pelo projeto CAD do suporte, e informação do fabricante do dispositivo
HVOF.
A Tabela~\ref{tab::massa_mh12} apresenta os resultados da massa estimada em cada
corpo:
%
\begin{table}[h]
\centering
\caption{Resultado do cálculo da massa de cada corpo}
\label{tab::massa_mh12}
\begin{tabular}{@{}clc@{}}
\toprule
\textbf{Corpo}                      & \textbf{Volume} [$m^3$] & \multicolumn{1}{l}{\textbf{Massa} [kg]} \\ \midrule 
Z 									& 0,01595         & 40,5							   \\
S                                   & 0,01443         & 36,7                               \\
L                                   & 0,00579         & 14,7                               \\
U                                   & 0,00997         & 25,3                               \\
R                                   & 0,00399         & 10,1                               \\
B                                   & 0,00100         & 2,5                                \\
T                                   & 0,00005         & 0,13                               \\
Ferramenta                          & -		          & 5,97							   \\ \midrule
\textbf{Volume total}~$=$           & 0,05119        & \multicolumn{1}{c}{$m^3$}	       \\
\multicolumn{1}{r}{\textbf{$\rho_m =$}} & 2540          &
\multicolumn{1}{c}{$kg/{m^3}$}     \\ \bottomrule
\end{tabular}
\end{table}
%

Para este modelo dinâmico, não será utilizada a última junta do robô, como foi
explicado na seção~\ref{sec::descricao_mh12}, de forma que pode-se considerar os
corpos ``B, T e Ferramenta'' como um único corpo. Assim, as propriedades de
massa e momento de inércia podem ser somadas e condensadas em uma única
descrição. 

A partir daqui, estes 3 últimos corpos serão tratadaos como um único corpo e
será utilizada a denominação do corpo B para representar esse conjunto.

Do CAD do MH12 também é possível extrair as as posições do centro de massa e
momentos de inércia de cada elo.
Como se considerou que a massa de cada elo é distribuida homogeneamente no seu
volume, isto resulta numa posição estimada do centro de massa. A
Tabela~\ref{tab::resumo_cm} resume os valores encontrados para a posição do
centro de massa com respeito ao referencial de cada elo e a
Figura~\ref{fig::pcm_mh12} ilustra no modelo CAD a posição aproximada.

\begin{figure}[h]
	\centering 
 	\includegraphics[width=0.7\textwidth]{figs/pcm_mh12}
 	\caption{Posição aproximada dos centros de massa}
 	\label{fig::pcm_mh12}
\end{figure}

%
\begin{table}[h] \centering
\caption{Posição do centro de massa}
\label{tab::resumo_cm}
\begin{tabular}{@{}cccc@{}}
\toprule
\textbf{Corpo} & \textbf{x [m]} & \textbf{y [m]} & \textbf{z [m]} \\ \midrule
Z              & 0,079      & -0,050     & 0,000      \\
S              & 0,124      & 0,027      & 0,008      \\
L              & 0,281      & -0,025     & -0,115     \\
U              & 0,123      & 0,115      & -0,025     \\
R              & 0,039      & -0,224     & 0,004      \\
B              & -0,011     & 0,146      & 0,000      \\ \bottomrule
\end{tabular}
\end{table}
%

O modelo CAD também calcula os momentos de inércia de cada elo obtidos no
centro de massa e alinhados ao sistema de coordenadas de referência daquele elo.
O resultado do cálculo dos momentos de inércia de cada elo são apresentados a
seguir.
%
\begin{align*}
InZ &=& \left( \begin {array}{ccc}  0,811& 0,293& 0,091\\ \noalign{\medskip}
 0,293& 0,695& 0,064\\ \noalign{\medskip} 0,091& 0,064& 0,923
\end {array} \right) & \quad InS &=& \left( \begin {array}{ccc}  0,811& 0,293& 0,091\\ \noalign{\medskip}
 0,293& 0,695& 0,064\\ \noalign{\medskip} 0,091& 0,064& 0,923
\end {array} \right) \\
InL &=&  \left( \begin {array}{ccc}  0,051&-
0,008&- 0,032 \\ \noalign{\medskip}- 0,008& 0,791& 0,011\\ \noalign{\medskip}- 0,032
& 0,011& 0,798\end {array} \right) & \quad InU &=& \left( \begin {array}{ccc} 
0,314& 0,149&- 0,077\\ \noalign{\medskip} 0,149& 0,367&- 0,057\\ \noalign{\medskip}- 0,077&- 0,057& 0,418
\end {array} \right) \\
InR &=&  \left( \begin {array}{ccc}  0,173&-
0,014& 0,001\\ \noalign{\medskip} - 0,014& 0,052& 0,004\\ \noalign{\medskip} 0,001& 0,004& 0,148
\end {array} \right) & \quad InB &=&  \left( \begin {array}{ccc}  0,235&- 0,018& 0,0\\ \noalign{\medskip}-
 0,018& 0,039& 0,0\\ \noalign{\medskip} 0,0& 0,0& 0,243\end {array}
 \right)
\end{align*}
%
No Sophia-Maple estes tensores são construidos com a função
\texttt{EinertiaDyad}, que aproveita o fato dos tensores de inércia serem
simétricos e recebe 6 argumentos para os termos da matriz e um para especificar
o sistema de coordenadas de referência em que foi obtido o tensor. Define-se
portanto cada tensor da seguinte forma:

\bigskip \noindent {\tt > for i from corpo[1] to corpo[6] do \\
\indent In||(corpo[i]):= EinertiaDyad( \\
\indent In||(corpo[i])||11, In||(corpo[i])||22, \\
\indent In||(corpo[i])||33, In||(corpo[i])||12, \\
\indent In||(corpo[i])||13, In||(corpo[i])||23, sc[i]) \\
end do:}

\subsubsection{Quantidade de Movimento e Quantidade de Movimento Angular}

Para o cálculo das forças de inércia, calcula-se a Quantidade de Movimento $G$ e
a Quantidade de Movimento Angular $H$, para cada corpo, de acordo com as
equações~\ref{eq::qntmov} e \ref{eq::qntmovang}.
%
\begin{align}
%eq1
	\mathbf{G}_{k} &= m_{k} \cdot \mathbf{v}_{k} \label{eq::qntmov} \\
%eq2	
	\mathbf{H}_{k} &= In_{k} \cdot \boldsymbol\omega^{k} \label{eq::qntmovang}
\end{align}
%

\subsubsection{Forças de Inércia}

As forças de inércia são a derivada temporal das quantidades de movimento, com
respeito ao referencial inercial Z, conforme descrito pelas
equações~\ref{eq::finG} e \ref{eq::finH}.
%
\begin{align}
%eq1
	\dot{\mathbf{G}}_{k} &= \frac{^{Z}\mathrm{d}}{\mathrm{d} t} (m_{k} \cdot
	\mathbf{v}_{k}) \label{eq::finG} \\
%eq2	
	\dot{\mathbf{H}}_{k} &= \frac{^{Z}\mathrm{d}}{\mathrm{d} t} (In_{k} \cdot
	\mathbf{\boldsymbol\omega^{k}}_{k}) \label{eq::finH}	
\end{align}
%
No Sophia-Maple, obtém-se primeiro os vetores quantidade de movimento e, em
seguida, constrói-se o \texttt{Kvector} das quantidades de movimento com a
função \texttt{KM} armazendado na variável \texttt{KMG}. Um função especial para
diferenciação de \texttt{Kvectors}, \texttt{Kfdt} é aplicada em \texttt{KMG} para
calcular os vetores das forças de inércia.

Vale notar que o Sophia-Maple adiciona um sufixo t aos termos que foram
derivados com respeito ao tempo.
Logo, as coordenadas generalizadas, q1,\ldots,qn, quando derivadas no tempo
resultam em q1t,\ldots,qnt. Este formato auxilirá a substituição destes termos
para solução das \textit{kinematic differential equations} (kde), conforme visto
na seção~\ref{sec::sophia-kane}.

O algoritmo para definir as Forças de Inércia do sistema no Sophia-Maple é o
seguinte:

\bigskip \noindent {\tt > for i from 1 to 6 do \\
\indent G||corpo[i]:= m||corpo[i] \&** v||corpo[i] \\
end do: \# Define as Quantidades de Movimento}

\medskip \noindent {\tt > for i from 1 to 6 do \\
\indent H||corpo[i]:= In||corpo[i] \&** w||corpo[i] \\
end do: \# Define as Quantidades de Movimento Angular}

\medskip \noindent {\tt > KMG:= subs(kde, KM[G||corpo, H||corpo]): \\
\# Define o Kvector das Quantidades de Movimento generalizadas}

\medskip \noindent {\tt > FIN:= subs(kde, Z Kfdt KMG): \\
\# Define o Kvector das Forças de Inércia}


\subsubsection{Forças Externas}

Uma das vantagens do método de Kane é não ser necessário avaliar as forças
internas, ou de restrição entre os corpos dos sistema. O motivo é que, ao se
projetarem as equações de equilíbrio dinâmico no hiperplano tangente, essas
forças desaparecem.

Logo, apenas as forças externas precisam ser modeladas para obter as equações de
movimento. No modelo do braço robótico, estas forças são o peso, os torques das
juntas e os torques de controle PID, em cada elo.

A força peso atua aplicada ao centro de massa de cada corpo \textit{k} e é calculada
pela equação~\ref{eq::peso}:
%
\begin{equation}
	\mathbf{Peso}_{k} = m_{k} \cdot \mathbf{g} \label{eq::peso}
\end{equation}
%
Onde $\mathbf{g}$ refere-se ao vetor da aceleração da gravidade, que atua sempre
no sentido de $-X$ no referencial inercial Z. É definido como:
%
\begin{equation}
	\mathbf{g} = [-9,81,~0,~0]^{Z}
\end{equation}
%
Os momentos externos são formados pelos torques dos atuadores em cada junta e
também os torques de controle PID. Neste momento é válido acrescentar uma
explicação rápida sobre a modelagem do controlador PID.

\subsubsection{Parâmetros de controle PID}

Para que o robô realize uma tarefa, é necessário seguir uma trajetória
previamente planejada. Uma trajetória pode ser simplesmente um conjunto de
posições de referência de cada junta, que varia no tempo, como foi abordado na
seção~\ref{sec::ikin_traj}. A variação das posição se dá pelo acionamento dos
torques em cada junta, que deve ser controlado para seguir a trajetória
planejada. 

Os fabricantes de robôs comerciais infelizmente não fornecem detalhes sobre o
método nem os parâmetros de controle utilizados.
Com o propósito de simular o controle do manipulador decidiu-se pela
simplicidade do método de controle PID.

Os parâmetros de controle PID são formados pelas variáveis proporcional,
integral e derivativa, ou $Kp$, $Td$ e $Ti$ respectivamente. Esses parâmetros
foram sintonizados para cada junta individualmente, fazendo-se simulações de
resposta para uma função degrau de referência. Os resultados mostraram que este
método é suficiente para seguir a trajetória e manter os erros dentro de uma
margem de tolerância aceitável. 
\todo[inline]{Incluir resultados da investigação dos parâmetros PID?}.

Como resultado da análise, encontrou-se os melhores resultados
para o conjunto de parâmetros apresentados na\todo{atualizar parâmetros PID}
Tabela~\ref{tab::pid}:
%
\begin{table}[h]
\centering
\caption{Parâmetros de controle PID em cada junta}
\label{tab::pid}
\begin{tabular}{@{}clll@{}}
\toprule
\textbf{Junta} & \textbf{Kp [Nm]} & \textbf{Ti [Nm]} & \textbf{Td [Nm]} \\ \midrule 
S              & 400         & 80000       & 3000        \\
L              & 2000        & 500000      & 15000       \\
U              & 1200        & 400000      & 12000       \\
R              & 100         & 100000      & 1000        \\
B              & 250         & 100000      & 1000        \\ \bottomrule
\end{tabular}
\end{table}
%

Logo, define-se as equações dos torques de controle para cada junta, em função
da diferença entre os parâmetros instantâneos e os de referência em cada junta.
O subíndice $k$ representa a lista de juntas, localizada na origem de cada
sistema de coordenadas de mesmo nome, variando em \{S,L,U,R,B\} e o
subíndice $i$ o número da coordenada generalizada associada.
%
\begin{equation}
	PID_{k} = Kp(q_i-q_i{ref}) + Ti\int_{t1}^{t2} (q_i-q_i{ref})dt +
	Td(u_i-u_i{ref}) \label{eq::pid}
\end{equation}
%
No Sophia-Maple declara-se os torques de controle representados na
equação~\ref{eq::pid} da seguinte forma:

\bigskip \noindent {\tt > for k from 1 to 5 do \\
		 \indent PID\_||{corpo[k]}:= Kp*(q||k - q||k||ref) + Ti*int(q||k - q||k||ref,
		 t) + Td*(u||k - u||k||ref) \\
		 end do:}

\subsubsection{Momentos Externos}

Para obter os momentos externos que atuam em cada corpo, o primeiro passo é
definir os torques que atuam em cada junta. Para eliminar os momentos causados
pelas forças \texttt{Peso}, avaliou-se os torques das juntas com respeito ao
centro de massa de cada corpo. As equações~\ref{eq::tjuntasi} a
\ref{eq::tjuntasf} descrevem os torques externos e de controle que atuam em cada
junta.
\begin{align}
	\mathbf{TjZS} &= (TZS - PID_{S})\cdot[1,~0,~0]^Z \label{eq::tjuntasi} \\
	\mathbf{TjSL} &= (TSL - PID_{L})\cdot[0,~0,~1]^S \\
	\mathbf{TjLU} &= (TLU - PID_{U})\cdot[0,~0,~1]^L \\
	\mathbf{TjUR} &= (TUR - PID_{R})\cdot[0,~1,~0]^L \\
	\mathbf{TjRB} &= (TRB - PID_{B})\cdot[0,~0,~1]^B \label{eq::tjuntasf}
\end{align}
%
As equações~\ref{eq::mexi} a \ref{eq::mexf} calculam o somatório dos momentos
externos em cada corpo. Nota-se que o torque aplicado em qualquer junta causa
um par de torques de ação e reação nos elos adjacentes. As variáveis $Mex_k$ são
grandezas vetoriais que representam o momentos externos resultante em cada
corpo.
%
\begin{align}
	\mathbf{Mex_{Z}} &= - \mathbf{TjZS } \label{eq::mexi}\\
	\mathbf{Mex_{S}} &= \mathbf{TjZS} - \mathbf{TjSL }\\
	\mathbf{Mex_{L}} &= \mathbf{TjLU }- \mathbf{TjSL }\\
	\mathbf{Mex_{U}} &= \mathbf{TjUR }- \mathbf{TjLU }\\
	\mathbf{Mex_{R}} &= \mathbf{TjRB }- \mathbf{TjUR }\\
	\mathbf{Mex_{B}} &= - \mathbf{TjRB} \label{eq::mexf}
\end{align}
%

A Figura~\ref{fig::sisfex} ilustra, em vista explodida, o equilíbrio das forças e
momentos externos em cada elo do manipulador.

\begin{figure}[h]
    \centering
    \begin{subfigure}[b]{0.6\textwidth}
        \includegraphics[width=\textwidth]{figs/forcas_ext}
        \caption{Forças externas em cada elo}
        \label{fig::fex}
    \end{subfigure}
    \quad %add desired spacing between images, e. g. ~, \quad, \qquad, \hfill
    % etc.
      %(or a blank line to force the subfigure onto a new line)
    \begin{subfigure}[b]{0.6\textwidth}
        \includegraphics[width=\textwidth]{figs/mom_ext}
        \caption{Momentos externos em cada elo}
        \label{fig::mex}
    \end{subfigure}
    \caption{Sistema de forças externas}\label{fig::sisfex}
\end{figure}

Note-se novamente que, pelo método de Kane, não é preciso introduzir as forças
internas, de restrição ou de contato entre os corpos, porque estas não
contribuem para as equações de movimento e desaparecem quando o sistema de
equações de equilíbrio dinâmico são projetads no hiperplano tangente, $\tau$.

No Sophia-Maple os vetores dos momentos externos resultantes são armazenadas em
\texttt{Evectors}. A sequência de equações~\ref{eq::tjuntasi} a \ref{eq::mexf}
torna-se:

\bigskip \noindent {\tt > \# Torques nas juntas} \\
		 \noindent {\tt > TjZS:= Evector(TZS - PID\_S,~0,~0,~S) } \\
		 \noindent {\tt > TjSL:= Evector(0,~0,~TSL - PID\_L,~L) } \\
		 \noindent {\tt > TjLU:= Evector(0,~0,~TLU - PID\_U,~U) } \\
		 \noindent {\tt > TjUR:= Evector(0,~TUR - PID\_R,~0,~R) } \\
		 \noindent {\tt > TjRB:= Evector(0,~0,~TUR - PID\_B,~B) }

\medskip \noindent {\tt > \# Momentos externos em cada elo} \\
		 \noindent {\tt > for k from 1 to 5 do \\
		 \indent Mex||k:= Tj||{k-1}||{k}  - Tj||{k}||{k+1} \\
		 end do:} \\
		 \noindent {\tt > MexB:= - TjRB:} \bigskip

		 
As forças e momentos externos podem ser agrupadas em um único vetor para
auxiliar a projeção destas em $\tau$. Novamente, faz-se uso da função
\texttt{KM}, o que retorna um \texttt{Kvector} contendo todas as forças e
momentos, descritos em cada sistema de referência.

\bigskip \noindent {\tt > FEX:= KM[Fex||corpo,~Mex||corpo]:}

		 
\subsubsection{Forças de Inércia e  Forças Externas Generalizadas}

Estas são as forças externas e de inércia FEX e FIN, quando projetadas no
hiperplano tangente, $\tau$. Para a projeção, faz-se uso da função
\texttt{\&kane}, do Sophia-Maple, o que retorna as equações implícitas
generalizadas de cada corpo:

\bigskip \noindent {\tt > FEXg:= tau \&kane FEX:} \\
		 \noindent {\tt > FINg:= tau \&kane FIN:}
		 
\subsubsection{Equações de Movimento}

Finalmente, as equações de movimento são formadas pelo equilíbrio dinâmico entre
as forças externas e externas generalizadas obtidas da projeção em $\tau$, de
acordo com a equação~\ref{eq::eqdin}:
%
\begin{equation}
	(FINg - FEXg)_{k} = 0 \label{eq::eqdin}
\end{equation}
%
O subíndice \textit{k} representa a lista de corpos \{Z,S,L,U,R,B\}. Como o
corpo Z está fixo no referencial inercial SC-Z, sua equação de movimento
torna-se a solução trivial $0=0$ e portanto, não fará parte do sistema. Obtém-se
então um conjunto de 5 equações de movimento, associadas aos 5 gdl do robô.

No Sophia-Maple atribui-se o conjunto de equações~\ref{eq::eqdin} à
variável \texttt{kane}, formando uma lista das equações de movimento. Portanto:

\bigskip \noindent {\tt > kane:= FINg - FEXg = 0} \bigskip

As equações denominadas \textit{kinematic differential equations}, ou
\texttt{kde}, completam o sistema de equações diferenciais que descrevem o
movimento do robô.
Para os 5 graus de liberdade do manipulador, tem-se 5 kde. As
equações~\ref{eq::kde5i} a \ref{eq::kde5f} descrevem o kde:
%
\begin{align}
	u_{1}(t) &= \frac{\mathrm{d} }{\mathrm{d} t}q_{1}(t) \label{eq::kde5i} \\
	u_{2}(t) &= \frac{\mathrm{d} }{\mathrm{d} t}q_{2}(t) \\
	u_{3}(t) &= \frac{\mathrm{d} }{\mathrm{d} t}q_{3}(t) \\
	u_{4}(t) &= \frac{\mathrm{d} }{\mathrm{d} t}q_{4}(t) \\
	u_{5}(t) &= \frac{\mathrm{d} }{\mathrm{d} t}q_{5}(t) \label{eq::kde5f}	
\end{align}
%
Somando-se esse sistema às equações de movimento e atribuindo à variável
\texttt{eq\_mov}, no Sophia-Maple:

\bigskip \noindent {\tt > eq\_mov:= kane union kde:} \bigskip

Logo, tem-se um sistema de 10 equações diferenciais ordinárias, não-lineares
(equações~\ref{eq::eqdin} a \ref{eq::kde5f}), cujas incógnitas são as
coordenadas generalizadas $q_{1}(t),\ldots, q_{5}(t)$ e as velocidades angulares
$u_{1}(t),\ldots, u_{5}(t)$.

Vale ressaltar que, para o sistema de apenas 5 equações de movimento simbólicas
associadas aos 5 graus de liberdade, as expressões são extensas demais para
serem representadas neste texto, o que ocuparia diversas páginas e obviamente
não traria nenhum valor à discussão deste tópico. Por isso, limita-se a
representação do sistema à equação~\ref{eq::eqdin}.

A função \texttt{cost} do Maple retorna o número de operações algébricas e de
funções de uma determinada equação ou sistema de equações fornecendo um
parâmetro para o usuário avaliar o tamanho das expressões e o custo
computacional do modelo.
Para se ter uma ideia, o custo do sistema de equações de movimento, representado
na forma puramente simbólica em \ref{eq::eqdin}, tem  a seguinte estrutura:
%
	$$ 18171\,{\it additions}+59804\,{\it multiplications}+39998\,{\it 
functions} $$
%
Porém, lembra-se que o sistema é puramente simbólico, porque até então não foram
substituidos os parâmetros numéricos do robô.
Este procedimento simplifica o sistema, porque muitos termos zerados
desaparecem.
Além disso, é possível fazer uso da função \texttt{simplify} do Maple, que é um
poderoso conjunto de rotinas para simplificação algébrica de expressões. Fazendo
uso desta função e da substituição dos parâmetros numéricos nas equações de
movimento, obtemos o novo custo algébrico das equações:
%
	$$ 5676\,{\it additions}+6992\,{\it multiplications}+1835\,{\it functions} $$
%
Nota-se a grande vantagem de realizar este procedimento, porque ao reduzir-se
drasticamente o número de termos e funções, reduz-se também o custo
computacional e por consequência o tempo para o cálculo da solução.

\subsubsection{Problema de Valor Inicial -- PVI}

Para resolver o sistema diferencial é necessário fornecer as condições de
contorno do problema. Como as equações estão no domínio do tempo, são fornecidas
as condições iniciais das variáveis de estado $q_{1}(0),\ldots, q_{5}(0)$ e
$u_{1}(t),\ldots, u_{5}(0)$.

Será adotado neste modelo que qualquer trajetória será iniciada a partir da
configuração incial do robô, ou seja:
%
\begin{equation}
	q_{1}(0) = \ldots = q_{5}(0) = u_{1}(0) = \ldots = u_{5}(0) = 0
	\label{eq::condini}
\end{equation}
%
O sistema completo será portanto formado pelas 10 equações de movimento
diferenciais e não lineares (\ref{eq::eqdin} a \ref{eq::kde5f}), com as 10
equações de valor incial definidas em~\ref{eq::condini}.

Para definir esse sistema no Sophia-Maple, primeiro cria-se a lista com as
equações de valor inicial e então adiciona-as ao sistema, utilizando a função
\texttt{union}:

\bigskip \noindent {\tt > cond\_ini:= \{seq(q||i(0) = 0, i=1..5), seq(u||i(0) =
0, i=1..5)\}} \\ 
		 \noindent {\tt > sistema:= eq\_mov union condini:} \bigskip

A solução deste sistema depende ainda da definição dos seguintes parâmetros:
%
\begin{itemize}
  \item{Torques nas juntas: TZS, TSL, TLU, TUR, TRB}
  \item{Coordenadas de referência dos ângulos de juntas: 
  		\\ $q_{1}ref, q_{2}ref, q_{3}ref, q_{4}ref, q_{5}ref$}
  \item{Velocidades angulares de referências das juntas: 
  		\\ $u_{1}ref, u_{2}ref, u_{3}ref, u_{4}ref, u_{5}ref$}
\end{itemize}
%
A trajetória efetuada pelo manipulador é então consequência da solução do
sistema, para um conjunto de parâmetros fornecidos. Para realizar uma trajetória
desejada, deve-se fornecer os parâmetros de referência e torques das juntas ao
longo do tempo que resultam naquela trajetória.

Se tal trajetória é descrita pela posição e orientação da ferramenta acoplada ao
último elo do manipulador, os parâmetros das juntas, tais que a ferramenta
segue aquela trajetória, devem ser determinados pelo cálculo da Cinemática
Inversa.


\subsection{Cinemática Inversa}\label{sec::ikin_mh12}

A solução da cinemática inversa se resume em encontrar os parâmetros das juntas
resume-se a encontrar os valores de $q_{1}, q_{2}, q_{3}$ que satisfazem a
posição do pulso e $q_{4}, q_{5}, q_{6}$ que satisfazem a orientação da
ferramenta. Na seção~\ref{sec::ikin} detalhou-se como encontrar uma solução
elegante pelo método geométrico, compatível para o manipulador MH12.

A pose final da ferramenta pode ser expressa em termos dos parâmetros
de posição e de orientação, de acordo com a equação~\ref{eq::posf}.

\todo{Incluir figura representando a posição e orientação final da ferramenta}
%
\begin{equation}
	\mathbf{x}_{f} = \begin{bmatrix}
		\mathbf{p}_{f} \\ \boldsymbol{\Phi}_{f}
	\end{bmatrix}
	\label{eq::posf}	
\end{equation}
%
Onde:
%
\begin{align}
\mathbf{p}_{f}^Z &= [x_f,y_f,z_f]^Z \\
\boldsymbol{\Phi}_{f}^Z &= [\phi,\theta,\psi]^Z
\end{align}
%



\subsubsection{Cálculo da posição do pulso}

A primeira tarefa será calcular a posição da ferramenta em termos das
coordenadas generalizadas.
Como visto na seção~\ref{sec::ikin_traj}, pode-se expressá-la convenientemente
pela posição da origem do pulso, porque esta só depende das coordenadas $q_1,
q_2$ e $q_3$.

A posição da ponta da ferramenta, $p_{f}$, é então a posição da origem do pulso,
$p_{w}$ em SC-R, somada ao vetor das dimensões da ferramenta, no referencial
SC-B:
%
\begin{equation}
	\mathbf{p}_{f}^Z = \mathbf{p}_{w} + [d_{x}, d_{y}, d_{z}]^B \label{eq::posf}
\end{equation}
%
Pelo desenvolvimento da cinemática direta, na seção~\ref{sec::dkin}, pode-se
expressar a posição de qualquer ponto da cadeia cinemática em um dado
referencial. Como o pulso tem origem no sistema de coordenadas SC-R, então:
%
\begin{equation} \label{eq::pwpr}
	\mathbf{p}_{w} =~^Z{\mathbf{p}}^R 
\end{equation}
%
Recorrendo às equações~\ref{eq::ikinq1} a \ref{eq::ikinq6}\todo{Incluir as
referências das equações}, desenvolvidas na seção~\ref{sec::ikin_traj}, pelo
método geométrico para a solução da cinemática inversa deste tipo de
manipulador, tem-se:
%
\begin{align*}
	q1 &= \atantwo (z_w,y_w) - \atantwo (d, \pm \sqrt{y_w^2 + z_w^2 - d^2}) \\
	q2 &= \atantwo (r,s) - \atantwo (\pm  \sqrt{1-E^2}, E) \\
	q3 &= \atantwo (\pm \sqrt{1-D^2}, D) - \frac{\pi}{2} + \atantwo (h, a3)
\end{align*}
%
Onde:
%
\begin{align*}
	D &= \frac{r^2 + s^2 -a2^2 - a3^2}{2 \cdot a2 \cdot a3} \\
	E &= - \frac{a3^2 - a2^2 - r^2 - s^2}{2 \cdot a2 \cdot \sqrt{r^2 + s^2}}
\end{align*}
%
Uma análise das expressões acima permite relacionar os termos geométricos do
robô com os parâmetros do MOTOMAN MH12. Substiui-se portanto estes termos de
acordo com as relações:
%
\begin{align}
	r &= \sqrt{x_w^2 + y_w^2 - d^2} - pSLy \\
	s &= x_w - (pZSx + pSLx) \\
	d &= 0 \\
	h &= pURx \\
	a2 &= pLUx \\
	a3n &= pURy \\
	a3 &= \sqrt{h^2 + a3n^2}
\end{align}
%
E obtém-se:
%
\begin{align}
	q1 &= \atantwo (z_w,y_w) \label{eq::q1ik} \\
	q2 &= \atantwo ( \sqrt {y_w^2 + z_w^2}- pSLy, x_w - pZSx - pSLx ) \nonumber \\
	   &- \atantwo ( \pm \sqrt {-E^2+1}, E ) \\
	q3 &= \atantwo ( \pm \sqrt {-{{\it D}}^{2}+1},{\it D} ) -\pi /2 +
\atantwo ( {\it pURx},{\it pURy} ) \label{eq::q3ik}
\end{align}
%
Onde:
%
\begin{align}
	D &= \textstyle \frac{1}{2}\,{\frac {-{{\it pLUx}}^{2}-{{\it pURx}}^{2}-{{\it
	pURy}}^{2}+ \left( \sqrt {{{\it y_w}}^{2}+{{\it z_w}}^{2}}-{\it pSLy} \right) ^{2
		}+ \left( {\it x_w}-{\it pZSx}-{\it pSLx} \right) ^{2}}{{\it pLUx}\,
		\sqrt {{{\it pURx}}^{2}+{{\it pURy}}^{2}}}} \\
	E &= \textstyle {-\frac{1}{2}\,{\frac {-{{\it pLUx}}^{2}+{{\it pURx}}^{2}+{{\it
		pURy}}^{2}- \left( \sqrt {{{\it y_w}}^{2}+{{\it z_w}}^{2}}-{\it pSLy} \right)
		^{2 }- \left( {\it x_w}-{\it pZSx}-{\it pSLx} \right) ^{2}}{{\it pLUx}\,
		\left[ \left( \sqrt {{{\it y_w}}^{2}+{{\it z_w}}^{2}}-{\it pSLy}
 		\right) ^{2}+ \left( {\it x_w}-{\it pZSx}-{\it pSLx} \right) ^{2}
 		\right]^{1/2}}}}
\end{align}
%
Determina-se portanto, pelo método geométrico, as coordenadas $q_1, q_2$ e
$q_3$ em função da posição do pulso e dos parâmetros geométricos do robô. Note que $x_w,
y_w, z_w$ são as coordenadas da posição do pulso, no referencial inercial e que
para calcular a posição da ferramenta, pela equação~\ref{eq::pwpr} precisa-se
ainda definir sua orientação.

Note-se também que existem 4 soluções para cada posição do pulso. Se
considerar-se os limites dos ângulos das juntas do robô, pode-se limitar o
número de soluções. Para exemplificar, a Figura~\ref{fig::elbowupdown} demonstra
duas soluções para a posição $\mathbf{p}_{w} = [0,7,~ 0,8,~ 0]^Z$: cotovelo para
cima em linha cheia e cotovelo para baixo em linha pontilhada.

\begin{figure}[h]
	\centering 
 	\includegraphics[width=0.75\textwidth]{figs/elbowupdown}
 	\caption{Exemplo de duas soluções possíveis para o mesmo ponto}
 	\label{fig::elbowupdown}
\end{figure}


\subsubsection{Cálculo da orientação da ferramenta}

Nesta etapa, deseja-se calcular a orientação da ferramenta, em relação ao
referencial inercial. Para isso são utilizados os ângulos $\phi,\theta,\psi$
para reresentar os ângulos de Euler das 3 rotações que orientam o referencial do
pulso na direção desejada.

Recorrendo novamente ao desenvolvimento da cinemática inversa geral para este
tipo de manipulador, da seção~\ref{sec::ikin_traj}, obteve-se da
equação~\ref{eq::rut} a relação:
%
\begin{equation*}
	R_{U}^{T} = (R_{Z}^{U})^T R_{yzy}
\end{equation*}
%
Onde $R_{zyz}$ representa a matriz rotação pelos ângulos de Euler
$\phi,\theta,\psi$, em torno dos eixos $Y,Z,Y$ respectivamente.

As matrizes de rotação $(R_{Z}^{U})^T$ e $R_{U}^{T}$ são calculadas para os
valores de $q_1, q_2$ e $q_3$ das equações~\ref{eq::q1ik} a \ref{eq::q3ik}.
Assim, a equação~\ref{eq::rutik} fornece um sistema de 9 equações e 6
incógnitas, se consideradas as funções senos e cossenos de $q_4, q_5$ e $q_6$ a serem
determinadas.
%
\begin{multline} \label{eq::rut}
%eq1
	 \left( \begin {array}{ccc} {\it c4}\,{\it c5}\,{\it c6}-{\it s4}\,{
\it s6}&-{\it c4}\,{\it s5}&{\it c4}\,{\it c5}\,{\it s6}+{\it c6}\,{
\it s4}\\ \noalign{\medskip}{\it s5}\,{\it c6}&{\it c5}&{\it s5}\,{
\it s6}\\ \noalign{\medskip}-{\it c5}\,{\it c6}\,{\it s4}-{\it c4}\,{
\it s6}&{\it s4}\,{\it s5}&-{\it c5}\,{\it s4}\,{\it s6}+{\it c4}\,{
\it c6}\end {array} \right) = \\
%eq2
	 \left( \begin {array}{ccc} {\it c2}\,{\it c3}-{\it s2}\,{\it s3}&-{
\it c2}\,{\it s3}-{\it s2}\,{\it c3}&0\\ \noalign{\medskip}{\it c1}\,{
\it c2}\,{\it s3}+{\it c1}\,{\it s2}\,{\it c3}&{\it c1}\,{\it c2}\,{
\it c3}-{\it c1}\,{\it s2}\,{\it s3}&-{\it s1}\\ \noalign{\medskip}{
\it s1}\,{\it c2}\,{\it s3}+{\it s1}\,{\it s2}\,{\it c3}&{\it s1}\,{
\it c2}\,{\it c3}-{\it s1}\,{\it s2}\,{\it s3}&{\it c1}\end {array}
 \right)^T \cdot \\
%eq3
	\cdot \left( \begin {array}{ccc} c\phi \,c\theta \,c\psi -s\phi \,s\psi &-c
\phi \,s\theta &c\phi \,c\theta \,s\psi +s\phi \,c\psi 
\\ \noalign{\medskip}s\theta \,c\psi &c\theta &s\theta \,s\psi 
\\ \noalign{\medskip}-s\phi \,c\theta \,c\psi -c\phi \,s\psi &s\phi \,
s\theta &-s\phi \,c\theta \,s\psi +c\phi \,c\psi \end {array} \right) 
\end{multline}
%

Analisando o lado esquerdo da equação~\ref{eq::rut} pode-se determinar
facilmente a solução para $\cos(q5)$, porque este será diretamente o resultado
alocado no termo $[2,2]$ da matriz 3x3 resultante do lado direito da equação, ou
seja:
%
\begin{multline}
\cos(q5) = {\it c2}\,{\it c3}\,s\phi \,{\it s1}\,s\theta -s\phi \,{\it s1}\,{\it 
		s2}\,{\it s3}\,s\theta +c\phi \,{\it c2}\,{\it s3}\,s\theta +
		\\ +c\phi \,{\it c3}\,{\it s2}\,s\theta +{\it c1}\,{\it c2}\,{\it c3}\,c\theta -{
		\it c1}\,{\it s2}\,{\it s3}\,c\theta 
\end{multline}
%
E seu seno é calculado por:
%
\begin{equation}
	\sin(q5) = \sqrt{1-\cos(q5)^2}
\end{equation}
%
Logo, utilizando a relação trigonométrica, $\sin(q5)/\cos(q5) = \tan(q5)$,
e invertendo a função tangente, determina-se:
%
\begin{equation}
	q5 = \atantwo(\sqrt{1-\cos(q5)^2}, \cos(q5))
\end{equation}
%

Para determinar $q4$ analisa-se novamente a matriz $R_{U}^{T}$. Se dividir o
termo $[3,2]$,  multiplicado por $(-1)$, pelo termo $[1,2]$, dos lados esquerdo
e direito da equação~\ref{eq::rut}, obtém-se uma expressão para $\tan(q4)$.
Invertendo a função tangente para obter $q4$:
\begin{multline} 
q4 = \atantwo( {{\it c1}\,s\phi \,s\theta -{\it s1}\,c\theta} , ~
 \left[  \left( s\phi \,{\it s1}\,{\it s3}-c\phi \,{\it c3} \right) {
\it c2}+{\it s2}\, \left( {\it c3}\,s\phi \,{\it s1}+c\phi \,{\it s3}
 \right)  \right] s\theta +
 \\ +{\it c1}\,c\theta \, \left( {\it c2}\,{\it 
s3}+{\it s2}\,{\it c3} \right) 
 )
\end{multline}
%
Similarmente à solução de $q4$, mas desta vez fazendo a divisão dos termos
$[2,3]$ por $[2,1]$, dos lados esquerdo e direito da equação~\ref{eq::rut},
encontra-se $q6$:
%
\begin{equation} \label{eq::q6ik}
	q6 = \atantwo ( {s6} , {c6} )
\end{equation}
%
Onde:
%
\begin{multline*}
	s6 = -{\it c2}\,{\it c3}\,s\phi \,s\psi \,{\it s1}\,c\theta +s\phi \,s\psi 
		\,{\it s1}\,{\it s2}\,{\it s3}\,c\theta +c\phi \,c\psi \,{\it c2}\,{
		\it c3}\,{\it s1}-c\phi \,c\psi \,{\it s1}\,{\it s2}\,{\it s3} +
		\\ -c\phi \,{\it c2}\,s\psi \,{\it s3}\,c\theta -c\phi \,{\it c3}\,s\psi \,{\it 
		s2}\,c\theta +{\it c1}\,{\it c2}\,{\it c3}\,s\psi \,s\theta -{\it c1}
		\,s\psi \,{\it s2}\,{\it s3}\,s\theta +
		\\ -c\psi \,{\it c2}\,s\phi \,{\it s3}-c\psi \,{\it c3}\,s\phi \,{\it s2}
\end{multline*}
\vspace{-15mm}
\begin{multline*}
	c6 = -c\psi \,{\it c2}\,{\it c3}\,s\phi \,{\it s1}\,c\theta +c\psi \,s\phi 
		\,{\it s1}\,{\it s2}\,{\it s3}\,c\theta -c\phi \,c\psi \,{\it c2}\,{
		\it s3}\,c\theta -c\phi \,c\psi \,{\it c3}\,{\it s2}\,c\theta 
		\\ -c\phi \,{\it c2}\,{\it c3}\,s\psi \,{\it s1}+c\phi \,s\psi \,{\it s1}\,{\it 
		s2}\,{\it s3}+c\psi \,{\it c1}\,{\it c2}\,{\it c3}\,s\theta -c\psi \,{
		\it c1}\,{\it s2}\,{\it s3}\,s\theta +
		\\ +{\it c2}\,s\phi \,s\psi \,{\it s3}+{\it c3}\,s\phi \,s\psi \,{\it s2}
\end{multline*}
%
Com o cálculo dos ângulos $q4, q5$ e $q6$ é possível determinar totalmente a
posição da ponta da ferramenta, pela equação~\ref{eq::posf}.
Como este ponto está alinhado ao eixo do efetuador, faz-se $dx = dz = 0$ e $dy
= pBTy$.
Reescrevendo a equação no referencial inercial, as componentes do vetor
$\mathbf{p}_{f}^{Z}$ são:\todo{Pensar em mudar estas eqs de lugar}
%
\begin{multline}
	x_f =  \left[  \left( -{\it c4}\,{\it s5}\,{\it pBTy}+{\it pURx} \right) {
		\it c3}+ \left( -{\it c5}\,{\it pBTy}-{\it pURy} \right) {\it s3}+{
		\it pLUx} \right] {\it c2} +
		\\ -{\it s2}\, \left( {\it c5}\,{\it pBTy}+{
		\it pURy} \right) {\it c3}+{\it s3}\, \left( {\it c4}\,{\it s5}\,{\it 
		pBTy}-{\it pURx} \right) {\it s2}+
		\\ +{\it pZSx}+{\it pSLx}
\end{multline}
\vspace{-15mm}
\begin{multline}
	y_f =  \{ \left[  \left( -{\it c4}\,{\it s5}\,{\it pBTy}+{\it pURx}
 		\right) {\it c3}+ \left( -{\it c5}\,{\it pBTy}-{\it pURy} \right) {
		\it s3}+{\it pLUx} \right] {\it s2}+ 
		\\ + \left[  \left( {\it c5}\,{\it 
		pBTy}+{\it pURy} \right) {\it c3}-{\it s3}\, \left( {\it c4}\,{\it s5}
		\,{\it pBTy}-{\it pURx} \right)  \right] {\it c2}+{\it pSLy} \} {
		\it c1} +
		\\ -{\it s5}\,{\it s1}\,{\it s4}\,{\it pBTy}
\end{multline}
\vspace{-15mm}
\begin{multline}
	z_f =  \{  \left[ \left( -{\it c4}\,{\it s5}\,{\it pBTy}+{\it pURx}
 		\right) {\it c3}+ \left( -{\it c5}\,{\it pBTy}-{\it pURy} \right) {
		\it s3}+{\it pLUx} \right] {\it s2}+ 
		\\ + \left[ \left( {\it c5}\,{\it 
		pBTy}+{\it pURy} \right) {\it c3}-{\it s3}\, \left( {\it c4}\,{\it s5}
		\,{\it pBTy}-{\it pURx} \right)  \right] {\it c2}+{\it pSLy} \} {
		\it s1} +
		\\ +{\it c1}\,{\it s4}\,{\it s5}\,{\it pBTy}
\end{multline}



Tem-se portanto, pelo conjunto de equações~\ref{eq::q3ik} a \ref{eq::q6ik}, a
solução cinemática inversa do manipulador MOTOMAN MH12.


\subsection{Tarefa e trajetória} \label{sec::tarefa_traj}

As trajetórias para representar este processo serão simplificadas neste
trabalho, pela consideração de que as superfícies são planas e que pode-se
representar a trajetória totalmente naquele plano.

A trajetória será modelada pelo conjunto de funções das posições de referência
para cada coordenada generalizada. Como o planejamento é feito pela função da
posição da ferramenta no tempo, são utilizadas as soluções da cinemática
inversa, desenvolvidadas na seção~\ref{sec::ikin_mh12}.

É claro que dependendo da posição e orientação da superfície a ser coberta pelo
robô, este assumirá diferentes configurações, que causam excitações diferentes
em sua base. Na seção~\ref{sec::casos}, diversos casos de trajetória e bases
serão analisados, com o objetivo de verificar os efeitos também da trajetória
nos erros devido a dinâmica do sistema robô-base. 

Para exemplificar os conceitos abordados nesta seção, considere-se a seguinte
tarefa para cálculo da trajetória:
%
\newline
\begin{tcolorbox}
[colframe=black!75!white, colback=white, title = Trajetória -Exemplo] 
  \textbf{Tarefa:} Revestimento por HVOF \\
  \textbf{Área:} $(600 \times 300)~mm^2$ \\
  \textbf{Ponto inicial:} $\mathbf{p}_f = [0,700,~1,238,~-0,600]^Z$ \\
  \textbf{Orientação:} $\boldsymbol{\Phi}_{f} = [0,~0,~0]^Z$ \\
  \textbf{Passo:} $10~mm$ \\
  \textbf{Número de paralelos:} 50 \\
  \textbf{Velocidade da ferramenta:} $40~m/min$
\end{tcolorbox}
%
De acordo com o exemplo, a orientação do plano da superfície,
está alinhado ao sistema de coordenadas inercial e paralelo ao plano xz. A
Figura~\ref{fig::trajec_600x500x10} apresenta os 50 paralelos da trajetória ao
longo desta região.

\begin{figure}[h]
	\centering 
 	\includegraphics[width=0.70\textwidth]{figs/trajec_600x500x10}
 	\caption{Exemplo da trajetória de revestimento de uma região}
 	\label{fig::trajec_600x500x10}
\end{figure}

A Figura~\ref{fig::trajec3D_600x500x10} apresenta essa tarefa no ambiente 3D
de simulação, com o robô no início da trajetória.

\begin{figure}[h!]
	\centering 
 	\includegraphics[width=0.50\textwidth]{figs/trajec3D_600x500x10}
 	\caption{Trajetória no ambiente 3D}
 	\label{fig::trajec3D_600x500x10}
\end{figure}

\subsubsection{Caminho ponto-a-ponto e função da trajetória}

O objetivo é determinar a função da trajetória em que robô realiza a tarefa de
revestimento. Primeiramente, define-se o conjunto ponto-a-ponto que forma o
caminho. A função temporal que liga os pontos do caminho é a trajetória.

Os pontos serão definidos nos vértices em que há mudança de direção da
ferramenta, ou seja, no início e fim de cada paralelo. Cada ponto será descrito
pela sua posição no ambiente em função do tempo. Logo:
%
\begin{equation} \label{eq::posi}
	\mathbf{p}_i = [~ px_{i}(t),~ py_{i}(t),~ pz_{i}(t)~]^Z
\end{equation}
%
Deseja-se que a velocidade da ferramenta entre os pontos seja controlada pelo
PID. Então, é preciso que a derivada da função trajetória entre os pontos seja a
velocidade desejada. Logo, a função da trajetória entre dois pontos de um mesmo
paralelo pode ser escrita como:
%
\begin{equation} \label{eq::pi}
\begin{split}
	p_i & = p_{i-1} + v \cdot \Delta t \\
		& = p_{i-1} + v \cdot (t-t_{i-1})
\end{split}
\end{equation}
%
Onde $v$ é a velocidade da ferramenta do ponto $p_{i-1}$ a $p_{i}$ e
$\Delta t$ é a variação de tempo entre o instante $t$ e o valor do tempo no
início da trajetória naquele paralelo $t_{i-1}$. A velocidade considerada neste
exemplo é de $40~m/min$.

Pode-se então calcular o intervalo de tempo entre cada ponto do caminho por:
%
\begin{equation} \label{eq::dt}
	\Delta t = \frac{p_i - p_{i-1}}{v}
\end{equation}
%
Portanto, o tempo associado a cada ponto do caminho é determinado por:
%
\begin{equation} \label{eq::ti}
	t_i = t_{i-1} + \Delta t
\end{equation}
%
Com as equações~\ref{eq::pi} e \ref{eq::ti}, define-se a função da trajetória
para cada paralelo.
%
\begin{table}[h]
\centering
\caption{Resumo dos pontos e funções trajetória do exemplo}
\label{tab::func_paralelos}
\begin{tabular}{|c|c|c|c|c||c|c|c|}
\hline
i & $px_i$ & $py_i$ & $pz_i$ & $t_i$  & $px_i(t)$    & $py_i(t)$ &
$pz_i(t)$
\\ \hline \hline
1       & 0,70  & 0,8   & -0,6  & 1,000 & 0,7         & 0,8      & -0,6         \\ \hline
2       & 0,70  & 0,8   & 0     & 1,900 & 0,7         & 0,8      & -1,27+0,67t  \\ \hline
3       & 0,71  & 0,8   & 0     & 1,915 & -0,57+0,67t & 0,8      & 0            \\ \hline
4       & 0,71  & 0,8   & -0,6  & 2,815 & 0           & 0,8      & 1,27-0,67t   \\ \hline
5       & 0,72  & 0,8   & -0,6  & 2,830 & -1.167+.67t & 0,8      & -0,6         \\ \hline
6       & 0,72  & 0,8   & 0     & 3,730 & 0,72        & 0,8      & -2,487+0,67t \\ \hline
7       & 0,73  & 0,8   & 0     & 3,745 & -1,77+0,67t & 0,8      & 0            \\ \hline
8       & 0,73  & 0,8   & -0,6  & 4,645 & 0,73        & 0,8      & 2,497-0,67t  \\ \hline
\end{tabular}
\end{table}
%

A Tabela~\ref{tab::func_paralelos} apresenta os termos do vetor posição da
equação~\ref{eq::posi} e as funções trajetória para os primeiros 4 paralelos do
exemplo. A Figura~\ref{fig::pontos_exemplo} ilustra os pontos indicados na
tabela no gráfico da trajetória.

\begin{figure}
	\centering 
 	\includegraphics[width=0.90\textwidth]{figs/pontos_exemplo}
 	\caption{Os pontos do caminho dos 8 primeiros paralelos}
 	\label{fig::pontos_exemplo}
\end{figure}

\subsubsection{\textit{Piecewise function} e cinemática inversa}

As funções $px_i(t), py_i(t)$ e $pz_i(t)$, podem ser tratadas como uma única
função, através do uso do comando \texttt{piecewise}, no Maple. Os argumentos
deste comando são simplesmente as funções contidas em cada intervalo de tempo,
resultando em uma função única, dividida por partes, sem necessidade de serem
contínuas. A vantagem é que essas funções podem ser derivadas, sendo reconhecido
que na discontiuidade a função é indefinida.
Outra vantagem é que pode ser utilizada diretamente para calcular, por meio da
cinemática inversa, as funções das coordenadas generalizadas que descrevem o
conjunto de posições, para todo o intervalo de tempo.

No Maple, a função \textit{piecewise} resultante tem a seguinte forma:
%
\begin{equation}
p_{piecewise} = 
\begin{cases}
p0 & t<t1 \\
p1 & t<t2 \\
p2 & t<t3 \\
p3 & t<t4 \\
p4 & otherwise
\end{cases}
\end{equation}
%

Pelas equações~\ref{eq::q3ik} a \ref{eq::q6ik} foram definidas as coordenadas
generalizadas $q1$ a $q6$ em funcão da posição da ferramenta, $\mathbf{p}_f^Z$.
Logo, basta substituir a função \textit{piecewise} de cada termo de
$\mathbf{p}_f^Z$ em $q1$ a $q6$ para obter o resultado da cinemática inversa por
partes, ou intervalos de tempo.

Esta nova função será utilizada como parâmetro de entrada no modelo dinâmico,
para as coordenadas de referência $q_1ref$ a $q_5ref$. Isso faz com que o
controle PID forneça os torques para atingir estas coordenadas ao
longo do tempo e assim perseguir a trajetória determinada. A
Figura~\ref{fig::res_exemplo} apresenta o gráfico do resultado da cinemética
inversa utilizando as função \textit{piecewise} para a posição e velocidade da
ferramenta durante os primeiros 8 paralelos da tarefa. 

\begin{figure}[h]
    \centering
    \begin{subfigure}[b]{0.40\textwidth}
        \includegraphics[width=\textwidth]{figs/qxt_exemplo}
        \caption{Resultado dos ângulos de referência cinemáticos}
        \label{fig::qxt_exemplo}
    \end{subfigure}
    \quad %add desired spacing between images, e. g. ~, \quad, \qquad, \hfill
    % etc.
      %(or a blank line to force the subfigure onto a new line)
    \begin{subfigure}[b]{0.4\textwidth}
        \includegraphics[width=\textwidth]{figs/uxt_exemplo}
        \caption{Resultado das velocidades de referência cinemáticos}
        \label{fig::uxt_exemplo}
    \end{subfigure}
    \caption{Resultados cinemáticos do exemplo de
    trajetória}\label{fig::res_exemplo}
\end{figure}


\subsection{Modelo teórico MBS -- Robô}

Nas seções anteriores foram desenvolvidas as equações simbólicas da
cinemática direta, cinemática inversa e dinâmica do manipulador robótico. 
Gerou-se portanto um modelo dinâmico (MBS) do robô, tal que é
possível realizar simulações de qualquer tarefa que se desejar. Para isso, são
listadas as seguintes etapas:

\begin{enumerate}
  \item Definir os sistemas de coordenadas locais dos elos do manipulador;
  \item Definir as coordenadas generalizadas;
  \item Fornecer os parâmetros de geometria e inércia do robô;
  \item Definir a configuração inicial do robô;
  \item Definir os parâmetros do controlador PID;
  \item Fornecer os requisitos da tarefa (no caso do revestimento: orientação,
  velocidade, passo);
  \item Definir os pontos do caminho a ser percorrido pela ferramenta;
  \item Calcular a trajetória e fornecer as coordenadas de referência no
  modelo dinâmico;
  \item Rodar a simulação da tarefa. 
\end{enumerate}

Utiliza-se novamente o exemplo abordado na seção~\ref{sec::tarefa_traj} para
demonstrar os resultados obtidos pelo modelo MBS. Dada a tarefa de revestimento
do exemplo, realiza-se a simulação do modelo dinâmico com os valores de
referência para o PID fornecido pela cinemática inversa.  Os resultados serão
apresentados para os primeiros 4 paralelos, a fim de facilitar sua visualização.

Este modelo MBS conta com uma animação do movimento
do robô realizando a trajetória, onde a posição da ferramenta deixa um rastro a
partir do início da trajetória planejada. A sequência de
Figuras~\ref{fig::mbs3D_exemplo} fornece uma visulização simplificada da
animação, ``fotografada'' em vários instantes de tempo.

\begin{figure}[h]
    \centering
    \begin{subfigure}[b]{0.4\textwidth}
        \includegraphics[width=\textwidth]{figs/mbs3D_0s}
        \caption{$0~s$}
        \label{fig::mbs3D_0s}
    \end{subfigure}
    \quad %add desired spacing between images, e. g. ~, \quad, \qquad, \hfill
    % etc.
      %(or a blank line to force the subfigure onto a new line)
    \begin{subfigure}[b]{0.4\textwidth}
        \includegraphics[width=\textwidth]{figs/mbs3D_1s5}
        \caption{$1,5~s$}
        \label{fig::mbs3D_1s5}
    \end{subfigure}
    \quad %add desired spacing between images, e. g. ~, \quad, \qquad, \hfill
    % etc.
      %(or a blank line to force the subfigure onto a new line)
    \begin{subfigure}[b]{0.4\textwidth}
        \includegraphics[width=\textwidth]{figs/mbs3D_2s5}
        \caption{$2,5~s$}
        \label{fig::mbs3D_2s5}
    \end{subfigure}
    \quad %add desired spacing between images, e. g. ~, \quad, \qquad, \hfill
    % etc.
      %(or a blank line to force the subfigure onto a new line)
    \begin{subfigure}[b]{0.4\textwidth}
        \includegraphics[width=\textwidth]{figs/mbs3D_3s}
        \caption{$3~s$}
        \label{fig::mbs3D_3s}
    \end{subfigure}
    \caption{Animação da trajetória no ambiente 3D} \label{fig::mbs3D_exemplo}
\end{figure}

O próximo resultado compara a variação angular das juntas ideal, pelo cálculo da
cinemática inversa, e o efetivo, resultado da trajetória no modelo dinâmico.
A Figura~\ref{fig::qxt_ex_realxideal} apresenta em linha pontilhada o valor dos
ângulos ideais, e em linha cheia, os efetivos.

\begin{figure}[h]
	\centering 
 	\includegraphics[width=0.95\textwidth]{figs/qxt_ex_realxideal}
 	\caption{Comparação ideal x efetivo dos ângulos das juntas}
 	\label{fig::qxt_ex_realxideal}
\end{figure}

Outro resultado analisado é a posição efetiva da ponta da ferramenta, em
comparação com a idealizada na cinemática inversa. Esse resultado será
extremamente importante para avaliar os casos de diferentes bases do modelo
acoplado robô-base e permitirá verificar o critério de erros máximos admissíveis
para o processo. A Figura~\ref{fig::errop_exemplo} apresenta o resultado
comparativo entre as posições ideais e efetivas. Cada linha do gráfico
representa uma coordenada (x, y, z) do vetor posição da ferramenta.

\begin{figure}[h]
	\centering 
 	\includegraphics[width=0.95\textwidth]{figs/errop_exemplo}
 	\caption{Comparação ideal x efetivo da posição da ferramenta}
 	\label{fig::errop_exemplo}
\end{figure}

Nota-se que neste modelo MBS o robô é capaz de seguir com boa precisão a
trajetória informada. Para facilitar a visualização, a
Figura~\ref{fig::erros_exemplo} apresenta um gráfico do erro de cada coordenada
da posição, ao longo da trajetória. A linha pontilhada representa o erro
absoluto, ou seja, a norma do erro em cada direção.

\begin{figure}[h]
	\centering 
 	\includegraphics[width=0.95\textwidth]{figs/erros_exemplo}
 	\caption{Erro de posição da ferramante em cada direção}
 	\label{fig::erros_exemplo}
\end{figure}

Observa-se por este resultado que o erro de posicionamento, a partir do início
da região a ser revestida ($t>1s$), é da ordem de $10^{-1}~mm$ durante
apriximadamente $70\%$ do caminho de um paralelo. Próximo ao início e fim dos
paralelos, esse erro cresce, devido ao transitório de direção e sentido do
deslocamento, que não são acompanhados perfeitamente pela dinâmica do robô.
Estes erros chegam a uma ordem de $10~mm$ nestas regiões.

Da mesma forma que foi feito para a posição, pode-se verificar os erros de
velocidade da ferramenta. Para isso, deriva-se a a posição da ferramenta, com
respeito ao tempo e obtém-se o vetor velocidade. O resultado de cada componente
e do valor absoluto do erro para a tarefa do exemplo é apresentado na
Figura~\ref{fig::errovel_exemplo}.

\begin{figure}[h]
	\centering 
 	\includegraphics[width=0.95\textwidth]{figs/errovel_exemplo}
 	\caption{Erro de velocidade da ferramenta em cada direção}
 	\label{fig::errovel_exemplo}
\end{figure}

Os erros tolerados variam de acordo com os critérios definidos na
seção~\ref{sec::requisitos}\todo{Incluir seção com os requisitos do
revestimento} e serão analisados mais detalhadamente na
seção~\ref{sec::casos}.

Por último, calcula-se o erro de orientação, definido como a diferença
entre a orientação da superfície dada pela tarefa e obtida pela cinemática
inversa, e a orientação resultante da dinâmica.

É representada por uma matriz, determinada pela transformação homogênea entre o
referencial inercial, SC-Z, e o referencial localizado no pulso, SC-B, já que a
ferramenta é fixada neste último. Logo, pode-se comparar as duas matrizes, ideal
e efetiva, para cálculo do erro.

Uma maneira mais prática de representar o erro de orientação é por meio de um
escalar, que represente o ângulo de desvio da ferramenta entre o caso ideal e o
efetivo. Isto pode ser feito levando em consideração que estas matrizes são
formadas por vetores unitários dispostos nas colunas da matriz e formam uma
base ortonormal.\todo{Incluir teorema que calcula o angulo entre as bases}.

Seja a orientação ideal representada pela matriz $Mi$, que é função dos 3
ângulos de Euler $\phi,~\theta$ e $\psi$, e a matriz resultante da dinâmica por
$Me$, função do resultado de $q1(t),\ldots,q5(t)$, calcula-se:
%
\begin{equation}
	R = Me(q(t))~Mi(\phi,\theta,\psi)^{-1}
\end{equation}
%
Aplicando $R$ à equação~\ref{eq::angdesvio}, calcula-se o ângulo de desvio entre
as duas matrizes.
%
\begin{equation} \label{eq::angdesvio}
	\theta_{erro} = \arccos\left(\frac{tr(R)-1}{2}\right)
\end{equation}
%
Para a tarefa do exemplo, as matrizes $Me$ e $Mi$ e $R$ em, por exemplo,
$t=2~s$ são:
%
 \begin{gather*}
 % eq1
 Me = \left( \begin {array}{ccc}  0.99755& 0.0&- 0.069955
\\ \noalign{\medskip}- 0.00262& 0.99930&- 0.037325
\\ \noalign{\medskip} 0.069903& 0.037413& 0.99685\end {array} \right) \\
% eq2
 Mi =  \left( \begin {array}{ccc}  1~& 0~& 0~\\ \noalign{\medskip} 0~&
 1~& 0~\\ \noalign{\medskip} 0~& 0~& 1~\end {array} \right) \\
 % eq3
R =  \left( \begin {array}{ccc}  1.0&- 0.00000058757&- 0.0021389
\\ \noalign{\medskip}- 0.000079441& 0.99930&- 0.037415
\\ \noalign{\medskip} 0.0021374& 0.037415& 0.99930\end {array}
 \right)
\end{gather*}
%
Aplicando $R$ na equação~\ref{eq::angdesvio}, obtém-se o erro de orientação
neste instante de tempo.
%
\begin{equation*}
	\theta_{erro}(t=2~s) = 0.0374898~rad
\end{equation*}
%
O resultado do desvio de orientação da ferramenta, ao longo de toda a tarefa é
apresentado no gráfico da Figura~\ref{fig::orierro_exemplo}.

\begin{figure}[h]
	\centering 
 	\includegraphics[width=0.95\textwidth]{figs/orierro_exemplo}
 	\caption{Erro de orientação da ferramenta}
 	\label{fig::orierro_exemplo}
\end{figure}

Novamente, nesta tarefa do exemplo, não se está interessado no erro antes do
manipulador iniciar a trajetória dos paralelos do revestimento, ou seja, entre a
posição inicial e o ponto p1.
Analisando, portanto, o gráfico a partir de $1~s$ percebe-se um erro periódico,
maior devido às variações de direção, com amplitude máxima de aproximadamente
$0,038~rad$, ou $2,18^{\circ}$.

Apesar de não ser possível, apenas por este resultado escalar, afirmar a direção
onde o erro é maior, pode-se comparar com o gráfico do erro de posicionamento,
na Figura~\ref{fig::erros_exemplo}. É possível inferir que este erro está mais
relacionado com o atraso do posicionamento na direção $z$ e portanto, grande
parte do desvio de orientação está na direção vertical, $x$ de SC-Z.

Nesta seção foram explorados os resultados que simulações utilizando o modelo
MBS podem fornecer. Para isso, tomou-se como exemplo uma tarefa de cobertura de
uma superfície plana. Os métodos de cálculo dos erros de posicionamento,
velocidade e orientação da extremidade da ferramenta foram demonstrados e serão
utilizados para comparação de diferentes casos do modelo acoplado robô-base
flexível, realizando tarefas de revestimento de superfícies planas, que serão
propostos na seção~\ref{sec::casos}.






% -.~.-.~.-.~.-.~.-.~.-.~.-.~.-.~.-.~.-.~.-.~.-
\section{Modelo da base} \label{sec::base}

A base é definida, neste trabalho, como qualquer estrutura, à qual o manipulador
robótico é fixado, que forneça flexibilidade, em forma de graus de liberdade,
passivos das forças dinâmicas exercidas pelo movimento do robô. A flexibilidade
da estrutura causará desvios do movimento esperado e controlado do robô e cabe
ao método proposto tentar quantificar esses desvios e verificar se os requisitos
do processo serão satisfeitos. Para isso, é desenvolvido um modelo teórico da
base, para ser adicionado ao modelo do robô, formando um sistema MBS acoplado.

Nesta seção é demonstrado como uma estrutura construída com geometria,
materiais, restrições e infinitos graus de liberdade que seriam muito complexos
para serem descritos analiticamente em um modelo MBS, pode ser simplificado como
um sistema de massa-mola-amortecedor de 6 graus de liberdade.
Considera-se um conjunto de sistemas de coordenadas localizado no ponto teórico
onde é fixado o robô, na estrutura. A esse sistema, correspondem 3 movimentos de
translação $x,y,z$ e 3 movimentos de rotação $\theta_x, \theta_y, \theta_z$, que
formam as coordenadas generalizadas do modelo MBS.

Portanto, para realizar a simplificação é necessário obter uma matriz de rigidez
e uma matriz de amortecimento, associadas aos 6 gdl do corpo sobre a base. A
matriz de rigidez é obtida numericamente pela Análise por Elementos Finitos
(AEF) e a matriz de amortecimento pelo resultado experimental do ensaio de
vibrações seguido do tratamento dos resultados, detalhado na
seção~\ref{sec::experimento}.

Por fim é demonstrado como são desenvolvidas as equações que regem o
movimento do corpo acoplado a esse sistema massa-mola-amortecedor.
Foi utilizada a mesma metodologia do modelo do robô para desenvolvimento das
equações de movimento: pelo método de Kane e com uso das rotinas do programa
Sophia-Maple.

\subsection{Geometria e CAD}

A primeira etapa para modelagem da base é representar sua geometria em um modelo
CAD. Este modelo será utilizado para as simulações pela Análise por Elementos
Finitos, a fim de se obter a sua matriz de rigidez.

Para ilustrar o método, será aproveitada a base projetada e construída para
testes do robô, em laboratório, previstos no projeto EMMA. Consiste em uma
estrutura metálica, formada por perfis extrudados de alumínio estrutural. Sobre
a estrutura um par de trilhos paralelos permitem o movimento longitudinal da
placa onde é fixado o robô. As dimensões principais são 2 metros de comprimento,
0,8 metros de largura e 0,6 metros de altura. A
Figura~\ref{fig::estrut_modelo_fisico} apresenta uma fotografia da base de
testes.

\begin{figure}[h]
	\centering 
 	\includegraphics[width=0.70\textwidth]{figs/estrut_modelo_fisico}
 	\caption{Base de testes do robô}
 	\label{fig::estrut_modelo_fisico}
\end{figure}

O projeto desta base foi desenvolvido no programa SolidWorks, onde de forma
bastante detalhada, são incluídos os desenhos das peças customizadas, peças
comerciais, acessórios de união e fixadores mecânicos que compõe a montagem CAD
da estrutura. Este modelo é apresentado na Figura~\ref{fig::estrutCAD}, com os
principais componentes indicados, inclusive o robô.

\begin{figure}[h]
	\centering 
 	\includegraphics[width=0.75\textwidth]{figs/estrutCAD}
 	\caption{Modelo CAD detalhado da base e robô}
 	\label{fig::estrutCAD}
\end{figure}

Este modelo CAD, muito completo, poderia ser diretamente utilizado nas
simulações para obtenção da matriz de rigidez desta estrutura. No entanto, o
alto nível de detalhamento dos componentes, além da definição dos contatos e
restrições entre as peças, geraria um modelo de malha muito densa, não linear e
grande custo computacional. E o resultado teria muito informações irrelevantes
para esta análise, como por exemplo, tensões em elementos de fixação ou
deformação do rolamento do trilho.

Logo, o objetivo da simplificação é obter apenas os resultados que importam
na análise, e ignorar outros efeitos que não alteram significativamente
estes resultados. 

Para o modelo da base de testes, foram utilizados elementos unidimensionais, de
viga, para formar a estrutura e os trilhos. A união entre esses componentes foi
feita por elementos rígidos, representando os parafusos de fixação do trilho na
estrutura. A placa de fixação do robô foi representada também por elementos
unidimensionais rígidos, assumindo-se que sua rigidez é muito superior a rigidez
da estrutura, e o ponto de fixação do robô está localizado no centro desta
placa.
Os rolamentos do trilho foram substituídos por uma conexão rígida entre a placa
e os trilhos. Assim, a flexibilidade desta estrutura, no modelo simplificado
depende apenas das deformações dos elementos de viga que representam o trilho e
a estrutura. A Figura~\ref{fig::estrutFEA} apresenta o modelo simplificado para
a AEF, desenvolvida no programa Autodesk Simulation Mechanical.

\begin{figure}[h]
	\centering 
 	\includegraphics[width=0.75\textwidth]{figs/estrutFEA}
 	\caption{Modelo da base para AEF}
 	\label{fig::estrutFEA}
\end{figure}

\subsection{Análise por Elementos Finitos} \label{sec::aef}

Foi demonstrado na seção~\ref{sec::rigidez} um método analítico e sistemático
para obter a matriz de rigidez de uma estrutura em qualquer ponto de
interesse. Para estruturas simples, com poucos graus de liberdade e poucos
elementos, o método analítico se mostra bastante prático. Já para estruturas
mais complexas, com muitos elementos, conexões, restrições, graus de liberdade
e não linearidades, resolver pelo método analítico pode ser impraticável ou
desgastante.

Como deseja-se a rigidez da estrutura para apenas um ponto de interesse -- o
ponto onde virtualmente está fixado o robô -- propõe-se uma forma de, utilizando
os conceitos do método apresentado e simulações por AEF, extrair a matriz de
rigidez para este ponto. As simulações realizadas são do tipo lineares e
estáticas. A seguir são detalhadas as etapas para definir o modelo para AEF.

\subsubsection{Partes, elementos e propriedades}

O modelo é divido em estrutura, trilho, interfaces e placa, como foi ilustrado
na Figura~\ref{fig::estrutFEA} Para cada parte são definidos o tipo de elemento,
suas propriedades e material.

A estrutura é formada por perfis de alumínio estrutural. Esta é modelada como
elementos unidimensionais de viga. Isso significa que suas propriedades são
uniformes ao longo de apenas uma dimensão em todo o elemento, bastando fornecer
apenas suas propriedades de seção transversal. A
Figura~\ref{fig::sectrans_bosch} mostra o corte de seção de um elemento da
estrutura e suas propriedades. O material associado à estrutura é o Alumínio
EN-AW-6060. e as propriedades são incluidas no modelo, de acordo com a
Tabela~\ref{tab::prop_mat}.

\begin{figure}[h]
	\centering 
 	\includegraphics[width=0.90\textwidth]{figs/sectrans_bosch}
 	\caption{Seção transversal do elemento da estrutura}
 	\label{fig::sectrans_bosch}
\end{figure}

\begin{table}
\centering
\caption{Propriedades mecânicas dos materiais do modelo AEF}
\label{tab::prop_mat}
\begin{tabular}{@{}llcc@{}}
\toprule
\textbf{Propriedade}   &             & \textbf{EN-AW 6060} & \textbf{AISI 316} \\ \midrule
Densidade              & {[}g/cc{]}  & 2,70                & 7,90             \\
Módulo de Elasticidade & {[}GPa{]}   & 69,5                & 200               \\
Coeficiente de Poisson & {[}1{]}     & 0,34                & 0,29    			\\
Tensão de escoamento   & {[}MPa{]}   & 150                 & 240               \\
Resistência à tração   & {[}MPa{]}   & 150                 & 580               \\ \bottomrule
\end{tabular}
\end{table}

O trilho também é modelado como elemento de viga. Sua seção transversal e
propriedades estão na Figura~\ref{fig::sectran_trilho}. Seu material é o aço
carbono inoxidável AISI 316 e suas propriedades mecânicas estão na
Tabela~\ref{tab::prop_mat}.

\begin{figure}[h]
	\centering 
 	\includegraphics[width=0.90\textwidth]{figs/sectran_trilho}
 	\caption{Seção transversal do elemento do trilho}
 	\label{fig::sectran_trilho}
\end{figure}

As partes interface e placa de fixação do robô são modelados como elementos
unidimensionais rígidos, ou seja, transmitem apenas forças e momentos, mas não
se deformam. Ao todo são 40 peças de interface representando os 20 parafusos de
fixação de cada trilho. Já a placa de fixação é formada por um quadro de
elementos ligando os dois trilhos e uma ``pirâmide'' convergindo para o ponto de
fixação do robô. Como este ponto representa a conexão entre a estrutura e o
robô, aplica-se os esforços na estrutura diretamente por ali. Os elementos
rígidos transmitem estes esforços ao trilho e à estrtura e obtém-se os resultado
de forças de reação ou deslocamentos resultantes.


\subsubsection{Restrições e conexões}

Esta base é originalmente fixa no piso por meio de parafusos chumbadores. Cada
perna da estrurura posssui 3 parafusos. No modelo AEF, esta restrição será
modelada como fixa, ou seja, não permite-se nenhuma translação ou rotação, e
é aplicada à extremidade inferior de cada perna.

Os elementos de cada parte são conectados por uniões ``soldadas'', ou seja, não
permitem nenhum grau de liberdade na extremidade do elemento onde há a conexão.


\subsubsection{Casos de Carregamento}

Para obter a matriz de rigidez, deve-se realizar ao todo 6 casos de carregamento
e restrições. Cada caso obtém 6 dos 36 termos, que correspondem a uma linha
da matriz 6x6. Os carregamentos e restrições são todos aplicados ao ponto de
fixação do robô.

A cada caso será aplicado um deslocamento prescrito unitário em uma direção. Às
outras direções é aplicada restrição à translação e à rotação. Logo, um
deslocamento unitário na direção $x$ ($ux = 1~mm$), por exemplo, provoca uma
força ou momento de reação em todas as direções. A força de reação
correspondente à direção x produz o termo de rigidez $k_{11}$, à direção y, $k_{12}$,
à direção z, $k_{13}$, e assim por diante.

A Tabela~\ref{tab::casoscarreg} resume os casos e restrições
simulados.

\begin{table}[h]
\centering
\caption{Resumo dos casos de carregamento e restrições}
\label{tab::casoscarreg}
\begin{tabular}{@{}ccccccc@{}}
\toprule
\textbf{Caso} & \textbf{ux} & \textbf{uy} & \textbf{uz} & \textbf{rx} & \textbf{ry} & \textbf{rz} \\ \midrule
\textbf{1}    & $0,1~mm$           & fixo        & fixo        & fixo        &
fixo & fixo        \\
\textbf{2}    & fixo        & $0,1~mm$            & fixo        & fixo        &
fixo        & fixo        \\
\textbf{3}    & fixo        & fixo        & $0,1~mm$            & fixo        &
fixo        & fixo        \\
\textbf{4}    & fixo        & fixo        & fixo        & $0,001~rad$           
& fixo        & fixo        \\
\textbf{5}    & fixo        & fixo        & fixo        & fixo        &
$0,001~rad$            & fixo        \\
\textbf{6}    & fixo        & fixo        & fixo        & fixo        & fixo    
& $0,001~rad$            \\ \bottomrule
\end{tabular}
\end{table}


\subsubsection{Malha}

Os elementos do trilho e da estrutura foram subdivididos para gerar a malha. A
malha controla a qualidade do modelo e precisão do resultado.
Quanto menor o elemento, ou mais discretizada a malha, menor são os erros
numéricos da solução. O processo de refinamento da malha, consiste em realizar
simulações iterativamente, com malha cada vez mais discretizada. A comparação
dos resultados a cada iteração permite julgar uma convergência da solução. Logo,
é realizado este procedimento,, até que a diferença do resultado seja
insignificante.

O resultado deste processo de refinamento gerou a seguinte malha:
%
$$ 49~parts,~ 620~nodes,~ 670~elements $$
%

\subsubsection{Resultados por AEF}

Os resultados obtidos pela análise estática são a distribuição de tensões,
deslocamentos, forças e momentos de reação em cada nó do modelo. Um exemplo do
resultado de tensões e deslocamentos é apresentado na Figura~\ref{fig::res_FEA},
para o caso de carregamento 6.

\begin{figure}[h]
    \centering
    \begin{subfigure}[b]{0.8\textwidth}
        \includegraphics[width=\textwidth]{figs/res_FEA_tensoes}
        \caption{Resultado AEF das tensões máximas resultantes}
        \label{fig::res_FEA_tensoes}
    \end{subfigure}
    \quad %add desired spacing between images, e. g. ~, \quad, \qquad, \hfill
    % etc.
      %(or a blank line to force the subfigure onto a new line)
    \begin{subfigure}[b]{0.8\textwidth}
        \includegraphics[width=\textwidth]{figs/res_FEA_desloc}
        \caption{Resultado AEF dos deslocamentos resultantes}
        \label{fig::res_FEA_desloc}
    \end{subfigure}
    \caption{Resultado da simulação AEF para o caso 6}
    \label{fig::res_FEA}
\end{figure}

O resultado das forças e momentos de reação no ponto de fixação do robô é
apresentado na Tabela~\ref{tab::res_FEA_forcas}. 

\begin{table}[h!]
\centering
\caption{Resultado das forças e momentos de reação para o caso 6}
\label{tab::res_FEA_forcas}
\begin{tabular}{@{}ccccccc@{}}
\toprule
\textbf{Caso} & \textbf{Fx [N]} & \textbf{Fy [N]} & \textbf{Fz [N]} &
\textbf{Mx [Nm]} & \textbf{My [Nm]} & \textbf{Mz [Nm]} \\ \midrule \textbf{6}   
& -3,00E-6 & 2,50E+3    & 1,67E-7    & 1,12E-6    & -7,16E-7   & -9,76E+3   \\ \bottomrule
\end{tabular}
\end{table}


\subsection{Matriz de Rigidez}

Deseja-se que a base seja simplificada como um sistema massa-mola-amortecedor de
6 gdl e introduzida no modelo MBS como uma matriz 6x6, que promove deslocamentos
ao robô, em função do carregamento resultante dos seus movimentos.

A matriz de rigidez possui 36 termos, chamados de coeficientes de influência.
A nomenclatura e posição de cada termo na matriz, está relacionado à direção da
força ou momento aplicado. Assim, o termo $k_{32}$ por exemplo refere-se a
rigidez na direção $y$, devido a uma força na direção $z$.

A matriz de rigidez tem a seguinte forma:
\begin{equation}
K = \begin{pmatrix} 
    k_{11} & \dots 	& k_{16} \\
    \vdots & \ddots & \\
    k_{61} &        & k_{66} 
    \end{pmatrix}
\end{equation}
%
Com o resultado das análises por AEF de cada caso, pode-se então calcular termo
a termo da matriz de rigidez da base. Como os deslocamentos prescritos são
arbitrários, faz-se o seguinte cálculo para converter a força de reação no
coeficiente de influência da matriz de rigidez:
%
\begin{align} \label{eq::kij}
	k_{ij} = \frac{F_j}{u_{i}} \qquad &  i,j = 1,\ldots,6
\end{align}
%
Onde $u_i$ representa o valor do deslocamento prescrito e $F_j$ o valor da força
ou momento de reação em cada caso de carregamento. A equação~\ref{eq::kij}
normaliza, portanto, o coeficiente de influência para um valor unitário de
deslocamento.

Note-se que, por definição, a matriz de rigidez é sempre positiva
definida\todo{Incluir referência para esta citação}, ou seja, simétrica e com
todos os autovalores positivos.
Isso faz com que não seja necessário calcular todos os 36 coeficientes de influências, mas apenas 21. 

Utilizando os resultados das simulações AEF e a equação~\ref{eq::kij}, a matriz
de rigidez para a base de testes é:
%
\begin{equation*}
	K =
\begin{pmatrix}
941537	&	0	&	0	&	0	&	0	&	0 \\
0	&	130543	&	0	&	0	&	0	&	-24981 \\
0	&	0	&	262149	&	0	&	5757	&	0 \\
0	&	0	&	0	&	57019	&	0	&	0 \\
0	&	0	&	5757	&	0	&	119561	&	0 \\
0	&	-24981	&	0	&	0	&	0	&	97595 \\
\end{pmatrix}
\end{equation*}
%
Os valores da matriz são expressos em $kN/m$ para os termos relacionados à
rigidez de translação e $kNm/rad$ para os termos relacionados à rigidez
torsional.
Os termos que aparecem zerados não são exatamente zero, mas valores menores que 1 e, por
simplificação, foram considerados nulos.


\subsection{Matriz de Amortecimento}

O amortecimento será modelado como proporcional, ou de Rayleigh, como foi
explicado na seção~\ref{sec::amortecimento}. A matriz de amortecimento é
definida, pela equação~\ref{eq::amort_prop}, repetida a seguir:
%
\begin{equation*}
	C = \alpha M + \beta K
\end{equation*}
%
Logo, para definir totalmente a matriz de amortecimento da base, basta fornecer
os dois parâmetros, $\alpha$ e $\beta$.

Na seção~\ref{sec::param_mod} apresentou-se um método para, a partir dos dados
experimentais de amortecimento modal, obter os parâmetros $\alpha$ e $\beta$ que
melhor se ajustam ao modelo. Na seção~\ref{sec::experimento} é detalhado o
procedimento para se obter os valores experimentais de amortecimento e, a partir
do resultado, são calculados os parâmetros para obter a matriz de amortecimento.



\subsection{Modelo teórico MBS -- Base}

Equivalentemente ao realizado para o robô, a base será modelada como um
sistema dinâmico, cujas equações de movimento são desenvolvidas pelo método de
Kane.

O Sophia-Maple é novamente utilizado desde a construção do modelo cinemático até
a determinação das equações de movimento, exatamente como foi realizado o modelo
do robô. Como já foram apresentados os comandos e sintaxe do programa na
seção~\ref{sec::dkin}, não será repetido nesta seção, ciente de que os comandos
são equivalentes. No Anexo~\todo{Incluir código anexo} é fornecido o código
completo do modelo.

Para se chegar às equações de movimento, primeiramente, define-se os sistemas de
coordenadas e as coordenadas generalizadas do sistema.
As variáveis $q1, q2$ e $q3$ representam as 3 translações do ponto de fixação do
manipulador, em $x,y$ e $z$, respectivamente, em relação ao sistema de
coordenadas inercial, SC-R. As 3 rotações serão representadas pelas variáveis
$q4, q5$ e $q6$ e serão descritos nos sistemas de coordenadas SC-R1, SC-R2,
SC-R3, respectivamente.
A coordenada $q4$ representa um ângulo de rotacão do sistema de referência SC-R1
em relação ao sistema SC-R; $q5$ o ângulo entre SC-R2 e SC-R1; e $q6$ o ângulo
entre SC-R3 e SC-R2, formando uma cadeia cinemática de ângulos relativos. A
Figura~\ref{fig::schem_scbase} ilustra o sistema inercial SC-R e o sistema SC-R3
acoplado ao corpo sobre a base.

\begin{figure}[h]
	\centering 
 	\includegraphics[width=0.5\textwidth]{figs/schem_scbase}
 	\caption{Desenho esquemático dos sistemas de referência da base e inercial}
 	\label{fig::schem_scbase}
\end{figure}

Neste modelo é considerado apenas um corpo, descrito em SC-R e localizado pelo
vetor $\mathbf{p}^M$, função de $q1,q2$ e $q3$.
%
\begin{equation}
	\mathbf{p}^M = [q1,~q2,~q3]^R
\end{equation}
%
Logo, se não há deslocamento da base ($q1=q2=q3=0$) então estará coincidente com
o ponto zero, do referencial inercial.

Seguindo o cálculo da cinemática, obtém-se os vetores velocidades e acelerações
angulares, exatamente como realizado na seção~\ref{sec::dkin}, mas desta vez
apenas para o corpo $M$.
Portanto, a equação~\ref{eq::velocM} calcula a velocidade do corpo, pela
derivada temporal do vetor posição em relação a SC-R. E a equação,
\ref{eq::velangM} calcula o vetor velocidade angular de $M$, no referencial SC-R
em relação ao referencial SC-R3, que considera as 3 rotações $q4, q5$ e $q6$ da
base.
%
\begin{gather} 
%eq1	
	^{R}\mathbf{v}^{M} = [u1,~u2,~u3]^R \label{eq::velocM} \\
%eq2
	\boldsymbol{\omega}^{M} = ^{R}\boldsymbol{\omega}^{R3} = [{\it s5}\,{\it
	u6}+{\it u4},-{\it c5}\,{\it s4}\,{\it u6}+{\it c4}\,{ \it u5},{\it c5}\,{\it c4}\,{\it u6}+{\it s4}\,{\it u5}]^{R}
\label{eq::velangM}
\end{gather}
%

Em seguida é composto o vetor das Velocidades Generalizadas e calculado o
Hiperplano Tangente.
%
\begin{gather}
	\mathbf{v}_{gen} = [ \mathbf{v}^{M},~ \boldsymbol{\omega}^{M},~ 2] \\
	\boldsymbol{\tau} = [\boldsymbol{\tau}_1,~ \boldsymbol{\tau}_2,~ 2]
\end{gather}
%

O cálculo das equações dinâmicas se divide em definir o vetor das Forças
Externas generalizadas e o vetor das Forças de Inércia generalizadas, para, em
seguida, projetá-los no hiperplano tangente e obter as equações de movimento.

As forças externas são formadas pela força peso, do corpo $M$ e
pelas forças conservativas e não conservativas da base, provenientes da rigidez
e do amortecimento, respectivamente. Portanto:
%
\begin{equation}
	\mathbf{Peso}_M = m_M \cdot \mathbf{g}
\end{equation}
%
Onde, desta vez o vetor gravidade está descrito no referncial SC-R. 
%
\begin{equation}
	\mathbf{g} = [-9,81,~0,~0]^{R}
\end{equation}
%
O termo $m_M$ representa o valor da massa ``suspensa'' pela base. Neste tópico
de apresentação do modelo desacoplado da base, o valor desta massa será a massa
total do robô, $130~kg$.

As forças conservativas são calculadas pela matriz de rigidez da base. Esta
matriz, multiplicada pelo vetor das coordenadas generalizadas, fornece um
conjunto de 6 equações, que retornam a força elástica em função dos
deslocamentos do corpo.
%
\begin{equation}
%eq1
	\mathbf{Fk} = \mathbf{K} \cdot \mathbf{q}
\end{equation}
\begin{equation}
%eq2
	\mathbf{Fk} = \begin{pmatrix} 
    k_{11} & \dots 	& k_{16} \\
    \vdots & \ddots & \\
    k_{61} &        & k_{66} 
    \end{pmatrix} \cdot 
    \begin{pmatrix} 
    q1 \\ 
    \vdots \\ 
    q6 
    \end{pmatrix}
\end{equation}


Logo, a equação~\ref{eq::fki} calcula a força elástica $Fk$ na direção $i$
associada a cada coordenada generalizada $qi$:
%
\begin{equation} \label{eq::fki}
	Fk_i = \sum_{j=1}^{6} [K]_{i,j} \cdot q_j
\end{equation}
%
Onde $[K]_{i,j}$ é o elemento $i,j$ da matriz de rigidez $K$. Note-se que os
termos a partir da 4ª linha da matriz, na verdade calculam os momentos,
associados às rotações $q4, q5$ e $q6$ da base.
Porém, não será feita essa distinção, sendo considerados tanto as forças quanto
momentos, como forças generalizadas.

Equivalentemente, obtém-se as expressões para as forças não conservativas, mas
desta vez, multiplica-se a matriz de amortecimento $C$ pelo vetor $u$ das
velocidades. Logo:
%
\begin{equation} \label{eq::fci}
	Fc_i = \sum_{j=1}^{6} [C]_{i,j} \cdot u_j
\end{equation}
%

Então, pelo equilíbrio, as forças que agem sobre o corpo suspenso pela base são:
%
\begin{equation} \label{eq::fexm}
	\mathbf{Fex}_M = \mathbf{Peso}_M - \mathbf{Fk} - \mathbf{Fc}
\end{equation}
%
As forças externas genearlizadas são a projeção do vetor das forças externas da
equação~\ref{eq::fexm} no hiperplano tangente $\tau$.
%
\begin{equation}
	\mathbf{FEX}g = \mathbf{Fex}_M \cdot \boldsymbol{\tau}
\end{equation}
%

As forças de inércia são calculadas pelas equações~\ref{eq::finG} e
\ref{eq::finH}, para o corpo $M$. Aplicando ao modelo da base, obtém-se as
forças de inércia generalizadas, pela projeção do vetor forças de inércia no
hiperplano tangente $\tau$.
%
\begin{gather}
	\mathbf{Fin}_M = [\dot{\mathbf{G}}_{M},~ \dot{\mathbf{H}}_{M}] \\
	\mathbf{FIN}g = \mathbf{Fin}_M \cdot \boldsymbol{\tau}
\end{gather}
%
Finalmente, a equação de movimento do corpo $M$ é obtida pelo equilíbrio das
forças externas e de inércia generalizadas. Portanto, tem-se 6 equações de
movimento, associadas aos 6 gdl do sistema:
%
\begin{equation}
	\mathbf{FEX}g - \mathbf{FIN}g = \mathbf{0}
\end{equation}
%
O sistema \textit{kinematic differential equation} (kde) fornece mais 6
equações, relacionando $q$ e $u$. As condições iniciais $q1(0),\ldots,q6(0)$ e
$u1(0),\ldots,u6(0)$ completam o problema.

Assim como no modelo do robô, o problema consiste em solucionar um sistema de
equações diferenciais ordinárias, não-lineares, cujas incógnitas são as
coordenadas generalzadas $q1(t),\ldots,q6(t)$ e as velocidades
$u1(t),\ldots,u6(t)$. A solução portanto será numérica, assim como no modelo do
robô.

Para demonstrar o modelo, considere-se os seguintes parâmetros de rigidez,
amortecimento e condição incial do sistema:
%
\begin{itemize}
  \item{\textbf{Rigidez:} igual a da base de testes}
  \item{\textbf{Amortecimento:} $\alpha = 0;~ \beta = 10^{-5}$}
  \item{\textbf{Condição inicial:} $q1(0) = q2(0) = q3(0) = 0,1~mm$, $q4(0) =
  0,001~rad$, $q5(0) = q6(0) = 0$}
\end{itemize}
%


\subsubsection{Resultados do exemplo}

A Figura~\ref{fig::res_qbase_exemplo} apresenta o resultado dos deslocamentos da
base, dada a condição incial fora da sua configuração de equilíbrio.

\begin{figure}[h]
    \centering
    \begin{subfigure}[b]{0.8\textwidth}
        \includegraphics[width=\textwidth]{figs/q123_base_exemplo}
        \caption{Resultado MBS da base: translações q1, q2, q3}
        \label{fig::q123_base_exemplo}
    \end{subfigure}
    \quad %add desired spacing between images, e. g. ~, \quad, \qquad, \hfill
    % etc.
      %(or a blank line to force the subfigure onto a new line)
    \begin{subfigure}[b]{0.8\textwidth}
        \includegraphics[width=\textwidth]{figs/q456_base_exemplo}
        \caption{Resultado MBS da base: rotações q4, q5, q6}
        \label{fig::q456_base_exemplo}
    \end{subfigure}
    \caption{Resultados dos primeiros 500~ms do deslocamento da base}
    \label{fig::res_qbase_exemplo}
\end{figure}

Note-se que, antes do tempo final da simulação, o sistema parece estar em
equilíbrio. O resultado de cada coordenada generalizada em $t=0,5~s$ é:
%
\begin{align*}
	q1(0,5) &= -0,20839\cdot 10^{-5}~m \\
	q2(0,5) &= 4,8952\cdot 10^{-8}~m \\
	q3(0,5) &= 8,4009 \cdot 10^{-12}~m \\
	q4(0,5) &= -1,6031\cdot 10^{-10}~rad \\
	q5(0,5)	&= -1,6260\cdot 10^{-11}~rad \\
	q6(0,5) &= 1,7857\cdot 10^{-8}~rad
\end{align*}
%
Verifica-se por este resultado que, no equilíbrio, os deslocamentos são
desprezíveis para o robô, tal que os valores máximos são da ordem de
$10^{-2}~mm$ em $q1$ e $10^{-11}~rad$ em $q5$.








% -.~.-.~.-.~.-.~.-.~.-.~.-.~.-.~.-.~.-.~.-.~.-
\section{Ensaio Experimental} \label{sec::experimento}

Os parâmetros modais da base, associados aos modos e frequências naturais são
obtidos por meio de um ensaio de vibrações.
A estrutura utilizada para o experimento que foi apresentada na
Figura~\ref{fig::estrut_modelo_fisico}, é instrumentada com acelerômetros a fim
de se obter as acelerações associadas a cada um dos 6 graus de liberdade da
base, no ponto de interação com o robô. Um martelo instrumentado é utilizado
para excitar a base e obter as Funções de Resposta em Frequência (FRF's) de cada
grau de liberdade. Estes dados são utilizados para estimar os parâmetros modais:
modos de vibração, frequências naturais e amortecimentos.

Os resultados experimentais serão então utilizados para calcular os parâmetros
$\alpha$ e $\beta$ da matriz de amortecimento proporcional.


\subsection{Bancada experimental e Instrumentação}

A base de testes é utilizada para os propósitos do ensaio e instrumentada com
acelerômetros. Para ter consistência entre os modelos teórico e experimental,
deseja-se obter as acelerações de um ponto virtual, localizado no centro da
placa de fixação do robô. Este ponto representa a conexão teórica entre a base e
o manipulador, que restringe qualquer translação ou rotação relativa entre eles.
Definindo-se um referencial fixo a este ponto, qualquer deslocamento e rotação
deste referencial transfere o mesmo movimento para o manipulador. Logo, este
ponto está para o modelo experimental assim como $\mathbf{p}^M$ está para o
modelo teórico da base. Da mesma forma, o referencial local da placa é
equivalente ao referencial SC-R3 do modelo teórico.

Levando isto em consideração, é preciso que as medidas coletadas do experimento
reflitam os deslocamentos deste ponto.

\subsubsection{Acelerômetros}

São utilizados no total 7 acelerômetros uni-axiais piezoelétricos, fixados à
placa por bases magnéticas. A Figura~\ref{fig::acelerometos-base} apresenta a
distribuição dos acelerômetros na estrutura, onde são representados pelos
pequenos cilindros indicados, cuja direção de medição é a mesma do seu eixo, e o
sentido positivo sempre para fora da placa.

\begin{figure}[h]
	\centering 
 	\includegraphics[width=0.95\textwidth]{figs/acelerometos-base}
 	\caption{Dsitribuição dos acelerômetros na base}
 	\label{fig::acelerometos-base}
\end{figure}

Assim como foi assumido no modelo para AEF da base, a placa de fixação do robô é
muito rígida em relação à estrutura de perfis de alumínio. Ou seja, as
deformações desta peça serão despezíveis e portanto a distância entre quaisquer
dois pontos sobre a placa permanecerá fixa no referencial local.
Isto permite a distribuição dos acelerômetros em qualquer ponto da placa
assumindo as acelerações são consideradas devido a movimento de corpo rígido e
não deformações elásticas da peça. 

Isto implica, que pode-se calcular a aceleração, em cada uma das 6 direções do
ponto virtual, a partir de relações aritméticas simples entre os sensores $a1$ a
$a7$. As acelerações lineares são indicadas pela variável $a$ e as
rotacionais por $\alpha$ seguidas de subíndice indicando sua orientação. Logo:
%
\begin{align}
	a_x &= \frac{a2+a3}{2} \label{eq::acel_ax}\\
	a_y &= \frac{a4+a5}{2} \\
	a_z &= - \frac{a6+a7}{2} \\
	\alpha_x &= \frac{a5-a4}{r1} \\
	\alpha_y &= \frac{a2-a1}{r1} \\
	\alpha_z &= \frac{a1-a3}{r2} \label{eq::acel_alphaz}
\end{align}
%
Onde $r1$ é a distância entre os acelerômetros $a1$ e $a2$, na direção $z$; e
$r2$ a distância entre os acelerômetros $a6$ e $a7$, na direção $y$.

Note-se que que as acelerações lineares são calculadas pela média entre dois
acelerômetros na mesma direção.
Já as acelerações angulares pela diferença entre os dois acelerômetros na mesma
direção, dividida pela distância, $r1$ ou $r2$ entre eles. Considere-se agora
que cada equação crie um sensor virtual, localizado no centro entre o par
calculado. Se a placa é rígida, pode-se considerar então que tais acelerações
são as mesmas correspondentes às do ponto central
da placa.

Tem-se portanto um vetor das acelerações generalizadas medidas no referencial
local da placa, em que cada termo representa uma direção dos 6 gdl da base.
%
\begin{equation}
	\mathbf{a} = \left[ a_x, a_y, a_z, \alpha_x, \alpha_y, \alpha_z \right]
\end{equation}
%

\subsubsection{Martelo Instrumentado}

A excitação da estrutura é feita com um martelo instrumentado com sensor de
força, da fabricante PCB. Pelo tamanho e peso da estrutura a ser testada, é
recomendável um impacto com energia suficiente para excitar toda a estrutura, a
fim de obter os modos de vibração em uma faixa larga de frequência. A energia do
impacto pode ser calculada por:
%
\begin{equation}
	E = m \cdot v
\end{equation}
%
Infere-se que, para ter uma energia mais alta, ou aumenta-se a velocidade do
impacto, ou a massa do martelo. Como os modelos de martelo com massa baixa
($<1~kg$) não seriam capazes de excitar a estrutura adequadamente, optou-se por
um martelo de massa elevada. O modelo PCB-086D50 foi escolhido para o ensaio.
Este martelo possui massa de $5,5~kg$ e faixa de medição de $\pm 25~kN$.

A faixa de frequência excitada também é função do material da ponta do martelo.
A Figura~\ref{fig::carta_badwidth} apresenta a força do impacto no espectro das
frequências, obtida pela \textit{Fast Fourier Transform} (FFT) a partir do sinal
no tempo.
Percebe-se que para uma mesma força de pico do impacto, uma ponta mais dura
fornece uma faixa mais larga que quando usada uma ponta macia.

\begin{figure}[h]
	\centering 
 	\includegraphics[width=0.95\textwidth]{figs/carta_badwidth}
 	\caption[Espectro de frequência do impacto do martelo PCB-086D50]{Espectro de
 	frequência do impacto do martelo PCB-086D50. \\	Fonte: adaptado de~\cite{manualpcb} }
 	\label{fig::carta_badwidth}
\end{figure}


\subsubsection{Placas de aquisição, chassi e programas}

Para aquisição dos dados experimentais são utilizadas duas placas NI-9234, da
fabricante National Instruments. Cada placa possui 4 canais. Uma placa recebe 4
acelerômetros e a outra 3, mais o martelo instrumentado.

Um chassi portátitil modelo NI cDAQ-9178 com capacidade de até 8 placas
de aquisição controla a temporização, sincronização e transferência dos dados
para o computador, via USB. 

O programa SignalExpress fornece uma interface para registro, leitura gráfica,
análise e pós processamento dos dados adquiridos. Será utilizado para realizar
a média das aquisições, cálculo das acelerações em cada direção, FFT do impacto
do martelo e a Função Resposta em Frequência (FRF) das acelerações.

Obtidas as FRF's das acelerações em cada direção, os resultados de
magnitude e fase das FRF's são importados no programa ME'Scope VES, que oferece
ferramentas prontas para análise modal e obtenção dos modos, frequências
naturais e amortecimentos, de forma bastante prática.

A Figura~\ref{fig::foto_bancada} apresenta uma fotografia da bancada
experimental, montada no Laboratório de Acústica e Vibrações (LAVI/COPPE-UFRJ) e
a Figura~\ref{fig::diagrama_bancada} um diagrama esquemático, indicando os
sensores, a base e os módulos de aquisição, assim como as conexões entre eles..

\missingfigure[figheight=0.6\textwidth]{Fotografia da bancada}

\begin{figure}[h]
	\centering 
 	\includegraphics[width=0.95\textwidth]{figs/diagrama_bancada}
 	\caption{Diagrama da bancada experimental}
 	\label{fig::diagrama_bancada}
\end{figure}
\todo[inline]{Incluir fontes das figuras dos sensores}



\subsection{Aquisição e tratamento dos dados}

Os dados coletados pelos acelerômetros e o martelo instrumentado são enviados
através das placas e chassi de aquisição para tratamento, em tempo real no
SignalExpress. O pós-processamento no programa calcula as FRF's de cada grau de
liberdade e os resultados são exportados para o ME'Scope para realizar análise
modal e estimativa dos parâmetros modais. Segue o detalhamento de cada uma
destas etapas.

\subsubsection{Aquisição no SignalExpress}

Antes de serem adquiridos, deve-se primeiro preparar o ambiente no
programa SignalExpress. Nele é mapeado o canal que receberá os dados
de cada sensor, fornecidos os tipos de sensores e os valores de sensibilidade,
faixas de trabalho, taxa de aquisição e número de amostras.

Para evitar os fenômenos de \textit{aliasing}, a taxa de aquisição foi definida
em $50~kHz$\todo{Revisar valores}, longe o suficiente da frequência de
ressonância esperada da base, e para evitar o vazamento espectral (ou
\textit{leakage}), o número de amostras como o dobro da taxa, neste caso, $100$
mil. Isto garante uma janela de tempo de cada amostragem de 2 segundos. Como o
sinal é quase totalmente amortecido neste intervalo de tempo, não há necessidade
de utilizar uma função de janelamento (ou \textit{window}) neste sinal.
É utilizada a função \textit{Averaging} para calcular a média das amostragens,
pelo método RMS (\textit{Root Mean Square}).
Serão coletadas 5 amostragens por direção de estímulo, para processamento e
cálculo dos dados.

Em seguida, são definidas as acelerações do ponto central da base, que são
calculadas pelas equações~\ref{eq::acel_ax} a \ref{eq::acel_alphaz}. No
SignalExpress é utilizada a função de pós-processamento \textit{Formula}, que
permite realizar operações algébricas entre até 4 sinais. Logo, são definidas 6
funções que calculam as acelerações do ponto virtual central da base.

Por fim, é utilizada a função \textit{Frequency Response} do pacote NI Sound and
Vibration Assistant do SignalExpress. Esta função recebe  os sinais do
martelo e as acelerações em uma direção, e calcula as Funções de Transferência
entre as acelerações (saída) e a força de impacto (entrada).

\subsubsection{Função de Transferência}

Seja a Função de Transferência definida por uma matriz $H_{n \times n}$, formada
pelos termos $h_{ij}$. Cada termo representa uma FRF da aceleração de cada
direção $i$ como resposta do impulso do martelo na direção $j$. Logo, para este
modelo de base, $n$ é igual ao número de graus de liberdade, 6. Portanto 36
termos definem a Função de Transferência. Como a matriz é
simétrica,\todo{Incluir referência da matriz simétrica}, podem ser deteminados
apenas 21 termos.
%
\begin{equation}
H =
\begin{pmatrix}
    h_{11} & h_{12} & h_{13} & h_{14} & h_{15} & h_{16} \\
		   & h_{22} & h_{23} & h_{24} & h_{25} & h_{26} \\
		   & 		& h_{33} & h_{34} & h_{35} & h_{36} \\
		   & 		&  		 & h_{44} & h_{45} & h_{46} \\
		   & 		&  		 &  	  & h_{55} & h_{56} \\
      \multicolumn{2}{c}{\text{sim.}} & & &    & h_{66} 
\end{pmatrix}
\end{equation}
%

Note-se que para calcular os termos devido a rotação da base $h_{i4}$ a
$h_{i6}$, então seria necessário excitar um momento em cada direção angular
($r_x,r_y,r_z$) e obter . Entretanto, é impossível excitar, com o martelo, um
momento puro na base. Logo, estes termos não podem ser calculados diretamente.
Em compensação, o ensaio conta com aquisição simultânea de 6 sinais de saída
(acelerações) para cada sinal de entrada (impacto) e portanto classifica-se como
SIMO (\textit{Single Input Multiple Outputs}). A vantagem deste tipo de ensaio é
que para o caso em que o número de sinais é igual ao número de graus de
liberdade, obtem-se exatamente uma coluna inteira da matriz de Transferência. De
uma linha ou uma coluna completa, já há informação suficiente para  se
determinar os parâmetros modais. Este conceito é útil, pela incapacidade de
estimular os momentos puros na estrutura.

\subsubsection{Resultados da aquisição}

Para verificar os resultados no SignalExpress, são reservadas áreas para
plotagem dos seguintes gráficos:
%
\begin{itemize}
  \item Acelerações $a_x, a_y, a_z, \alpha_x, \alpha_y, \alpha_y$ vs.
  tempo;
  \item Força do martelo vs. tempo;
  \item FFT da força do martelo;
  \item FRF's: $a_x, a_y, a_z, \alpha_x, \alpha_y, \alpha_y$ para $F_x, F_y,
  F_z$: \\ Tal que em cada FRF fornece: Magnitude; Fase; parte Real; parte
  Imaginária; e Coerência.
\end{itemize}
%
Como são muitos, ao todo 27 gráficos, serão apresentados apenas os
resultados para o ensaio de impacto na direção $y$ da base.

\missingfigure{Resultado acelerações vs. tempo}

\missingfigure{Resultado Força do martelo vs. tempo}

\missingfigure{FFT da força do martelo}

\missingfigure{FRF: Magnitude}

\missingfigure{FRF: Fase}

\missingfigure{FRF: Real}

\missingfigure{FRF: Imaginário}

\missingfigure{FRF: Coerência}

A análise modal será feita no programa ME'Scope. Para isso, os resultados devem
ser exportados em arquivo ASCII na forma numérica, tal que a amostragem de dados
estjeda disposta em colunas, na seguinte ordem:
%
\begin{equation*}
\begin{matrix}
Frequencia & Magnitude~\#1  & Fase~\#1  & \ldots & Magnitude~\#6 & Fase~\#6 \\ 
 \vdots & \vdots & \vdots &  & \vdots & \vdots
\end{matrix}
\end{equation*}
%
As colunas de Magnitude e Fase correspondem aos
resultados das 6 FRF's relativas aos 6 graus de liberdade coletados.

\subsubsection{Análise modal no ME'Scope}

O programa ME'SCope é utilizado para identificar os parâmetros modais a partir
dos dados importados. A interface de importação do programa ajuda a identificar
os dados e relacionar às direções de medição e excitação correspondentes. 

A opção \textit{Modal Analysis} no ME'Scope é utilizada para caracterizar as
ressonâncias do sistema e estimar, por uma lista de FRF's fornecidas, os
parâmetros como frequências naturais, amortecimentos modais e modos de vibração.

O programa então faz um ajuste de curva (\textit{curve fitting}) das FRF's para
estimar os parâmetros modais. A primeira etapa é identificar a faixa de
frequência que se deseja analisar. Em seguida, informar quantos modos ou picos
há naquela faixa. O programa ainda permite a contagem automática, se fornecido um
valor mínimo para caracterizar o modo.

\missingfigure{Inlcluir imagem da contagem de picos}

Com isso o programa realiza uma estimativa dos coeficientes de um polinômio
do modelo FRF analítico, pelo ajuste de curva ao modelo experimental. Os
coeficientes são então processados para obter os parâmetros modais.

\missingfigure{Resultados da análise modal}

Os resultados da análise modal encontram-se na Tabela

\todo[inline]{Incluir tabela dos resultados da análise modal}.



O programa também permite construir o modelo experimental 3D com uma ferramenta
de modelagem interativa. Pontos específicos da malha que representa o modelo 3D
são relacionado aos dados fornecidos. Então é possível visualizar como a
estrutura se comporta dinamicamente por meio da animação do modelo, que de forma
interativa pode-se selecionar qualquer ponto na faixa de frequência e obter-se
uma animação do modo de vibração naquela frequência.

\missingfigure{Modelo 3D da base ME'Scope}



\subsection{Cálculo da Matriz de Amortecimento}

A matriz de amortecimento tem dimensão $6 \times 6$, que corresponde aos 6 graus
de liberdade (3 translações e 3 rotações) associados às 6 coordenadas
generalizadas do modelo teórico. Para facilitar seu uso e generalização é
utilizado o modelo de amortecimento proporcional, em que assume-se
que o amortecimento depende apenas de dois parâmetros, $\alpha$
e $\beta$, proporcionais às matrizes de inércia e rigidez respectivamente, de
acordo com a expressão:
%
\begin{equation*}
	C = \alpha  M + \beta  K
\end{equation*}
%

Como demonstrado na seção~\ref{sec::amortecimento}, os parâmetros $\alpha$
e $\beta$ se baseiam no resultado do amortecimento modal ao longo de uma
faixa de frequência de interesse. Repetindo a expressão:
%
\begin{equation*} \label{eq::alpha_beta}
	\xi_j = \frac{\alpha}{2 \omega_j} + \frac{\beta \omega_j}{2}
\end{equation*}
%
Onde $\xi_j$ é o amortecimento modal na frequência $ \omega_j$. Esses valores
são os obtidos experimentalmente, e estão na
Tabela~\ref{tab::resultado_modal}.
%
\begin{table}[h]
\centering
\caption{Resultado do amortecimento modal}
\label{tab::resultado_modal}
\begin{tabular}{@{}cccc@{}}
\toprule
Modo & Frequência & Amortecimento {[}Hz{]} & Amortecimento {[}\%{]} \\ \midrule
1    &            &                        &                        \\
2    &            &                        &                        \\
3    &            &                        &                        \\
4    &            &                        &                        \\
5    &            &                        &                        \\
6    &            &                        &                        \\ \bottomrule
\end{tabular}
\end{table}
%

Dos 6 modos de ressonância, tem-se o seguinte gráfico, na
Figura~\ref{fig::amort_x_freq}, do amortecimento em função da frequência, :

\missingfigure{Gráfico do amortecimento vs frequência}

Fazendo o ajuste do gráfico, em termos da equação~\ref{eq::alpha_beta},
obtém-se:
%
\begin{equation}
	\xi = \frac{48651518}{\omega} + 123546 ~\omega
\end{equation}
%
Logo, pode-se concluir que:
%
\begin{align}
	\frac{\alpha}{2} &= 48651518 \Rightarrow  \alpha = 8584654 \\
	\frac{\beta}{2} &= 123546 \Rightarrow  \beta = 213145
\end{align} \todo{Corrigir valores}
%

E finalmente, a matriz de amortecimento resultante, obtida experimentalmente é:
%
\begin{equation}
	C = 8584654~M + 213145~K
\end{equation}
%



% -.~.-.~.-.~.-.~.-.~.-.~.-.~.-.~.-.~.-.~.-.~.-
\section{Modelo acoplado robô e base} \label{sec::acoplado}

Demonstrado como forma construídos os modelos dinâmicos em separado, este
consiste em acoplar os dois modelos MBS. Isto é feito considerando que o sistema
de coordenadas SC-Z, inercial no modelo do robô, agora coincide com o sistema de
coordenadas final do modelo da base, SC-R3.

Note-se que no modelo da base, era considerado um único corpo, fixado no
``topo'' da cadeia cinemática de 6 graus de liberdade devido a elasticidade da
base.
Agora, este corpo é representado pelo pedestal do robô, corpo Z, seguido dos 5
elos consecutivamente conectados, tornando-o um MBS de 11 graus de liberdade. A
Figura~\ref{fig::esq_acoplado} ilustra o sistema acoplado, indicando as novas
coordenadas generalizadas associadas à base, $q1,\ldots,q6$, e associadas ao
robô, $q7,\ldots,q11$. O sistema de coordenadas SC-Z do manipulador aparece
coincidente com SC-R3 da base e o novo sistema de coordenadas inercial é
indicado por SC-R.

\begin{figure}[h]
	\centering 
 	\includegraphics[width=0.50\textwidth]{figs/esq_acoplado}
 	\caption[Novas coordenadas generalizadas do modelo MBS acoplado]{Novas
 	coordenadas generalizadas do modelo MBS acoplado \\
 	Fonte: adaptado de}
 	\label{fig::esq_acoplado}
\end{figure}
\todo[inline]{Incluir fonte nesta figura}

Para este sistema acoplado, é empregado o mesmo procedimento das
seções~\ref{sec::robo} e \ref{sec::base} para se deduzir as equações de
movimento, pelo Método de Kane, utilizando as rotinas do Sophia-Maple. A
principal diferença está na interface entre o robô e a base, em SC-R3/SC-Z.
Do ponto de vista do modelo do manipulador, o pedestal -- ou corpo Z -- que
antes não era considerado na dinâmica do sistema por estar fixo no referencial
inercial, agora irá sofrer os esforços externos, oriundos da conexão com a base
e também forças de inércia. O equilíbrio das forças externas na interface entre
os dois sistemas é apresentado no diagrama de corpo livre da
Figura~\ref{fig::dcl_interface}. Os outros elos do manipulador aparecem
sombreados, apenas como referência.

\begin{figure}[h]
	\centering 
 	\includegraphics[width=0.45\textwidth]{figs/dcl_interface}
 	\caption{Diagrama de corpo livre da interface entre o pedestal do robô e a
 	base}
 	\label{fig::dcl_interface}
\end{figure}

Verifica-se que a rigidez e amortecimento da base atuam como forças e momentos
externos no robô. A expressão que define as forças externas que atuam no corpo
Z, que no modelo desacoplado apresentava apenas o peso do corpo, agora é:
%
\begin{equation}
	\mathbf{Fex}_Z = \mathbf{Peso}_Z - \mathbf{Fk} - \mathbf{Fc}
\end{equation}
%
Da mesma forma, a nova expressão para os momemntos externos que atuam no corpo
Z, que antes considerava apenas o torque da junta ZS, torna-se:
%
\begin{equation}
	\mathbf{Mex}_Z = -\mathbf{TjZS} - \mathbf{Tk} - \mathbf{Tc}
\end{equation}
%

Estas expressões são perfeitamente equivalentes às apresentadas na
equação~\ref{eq::fexm}, que tratava o corpo sobre a base como uma massa M
qualquer. Portanto, é de se notar que as expressões já desenvolvidas para o
cálculo dos modelos desacoplados do robô e da base, são perfeitamente
equivalentes às do modelo acoplado, sendo necessário apenas fazer a mudança das
coordenadas generalizadas e corrigir a indicação dos corpos, para refletir o
novo modelo.

O aumento do número de graus de liberdade, de 6 para 11, aliado a um novo
conjunto de forças externas, causa um grande aumento de complexidade no
cálculo algébrico das equações de movimento. Para se ter uma ideia, o novo custo
do sistema de equações de movimento do modelo acoplado robô e base, tem a
seguinte estrutura.
%
$$ 72220~\textit{additions} + 617367~\textit{multiplications} + 409552~
\textit{functions} $$
%


\subsection{Robô sobre base rígida}

A base é considerada rígida quando seus deslocamentos devido a elasticidade de
sua estrutura causa erros desprezíveis de trajetória do robô. Ou seja erros
muito pequenos de posição, velocidade e orientação do seu efetuador. Uma base
rígida portanto seria um caso especial do modelo acoplado robô e base, tal que a
matriz de rigidez apresentasse valores muito elevados. No entanto, a medida que
se aumenta a rigidez da base, o modelo converge para o caso do sistema
desacoplado, no sentido que a rigidez é tão alta que os graus de liberdade
oferecidos pela base resultam em movimentos desprezíveis.

Entende-se que do cálculo da cinemática inversa, é definida a trajetória ideal,
que fornece os parâmetros de posição, velocidade e orientação exatos que se
deseja realizar.
Porém, a cinemática inversa fornece apenas um dado de entrada ao modelo
dinâmico, tal que a trajetória real será consequência da inércia do sistema e da
eficácia do controlador PID. A diferença entre a trajetória resultante e a
fornecida é a denominada erro de trajetória, onde os erros são apresentados em
termos de posição, velocidade e orientação. Logo, é de se esperar que, se o robô
é bem projetado e seu controle bem determinado, os erros apresentados estarão
dentro de uma faixa de erro aceitável. Na seção~\ref{sec::hvof} foram
apresentados erros tolerados para o processo de revestimento por asperção
térmica. Este processo será utilizado para exemplificar a utilização do método
para alguns casos de bases flexíveis, na seção~\ref{sec::casos}.

O modelo do robô sobre base rígida será portanto utilizado para descrever uma
trajetória de referência, ou seja, resultado puramente da dinâmica do robô,
desconsiderando qualquer grau de liberdade da base. Seus resultados servirão de
referência, portanto, para comparar com os do manipulador sobre bases flexíveis.

\subsection{Robô sobre base flexível}

Quando a base possui pouca rigidez, e havendo deslocamentos consideráveis devido
à dinâmica do robô, o modelo acoplado deve ser utilizado. A mesma trajetória do
modelo desacoplado, é fornecida e obtém-se os novos resultados. Calcula-se então
os erros de posição, velocidade e orientação, e compara-se com os resultados do
outro modelo.




% -.~.-.~.-.~.-.~.-.~.-.~.-.~.-.~.-.~.-.~.-.~.-
\section{Casos aplicados}\label{sec::casos}

Será apresentada a aplicação do método para alguns casos propostos. Primeiro,
são definidas 2 trajetórias, diferentes em requisitos de velocidade, passo e
orientação. Em seguida, estas trajetórias são aplicadas ao modelo de base
rígida, que fornece os resultados de referência, e simuladas no modelo acoplado
à base flexível. Os casos de base são analisados para 3 versões, da mais rígida
à mais flexível, cujos modelos são detalhados a seguir. Os resultados são
analisados e discutidos no Capítulo~\ref{cap::resultados}.

De maneira geral, o método para análise de casos segue o seguiunte fluxo de
tarefas:
%
\begin{enumerate}
  \item Definir a cinemática direta do manipulador robótico utilizado;
  \item Definir a trajetória, e calcular sua função pela cinemática inversa;
  \item Calcular a matriz de rigidez da base por Análise por Elementos Finitos;
  \item Calcular a matriz de amortecimento da base;
  \item Fornecer a função da trajetória ao MBS acoplado;
  \item Simular e verificar os resultados.
\end{enumerate}

Em todos os casos, o manipulador utilizado é o MOTOMAN MH12, que foi definido
no modelo do robô. As seções a seguir definem as trajetórias e os parâmetros
de rigidez e amortecimento dos casos de bases propostos.

\subsection{Trajetórias do efetuador}

Para exemplificar o uso do método, são propostas 2 trajetórias de simulação de
revestimento de uma superfície plana, em zigue-zague. Este tipo de trajetória
corresponde a uma tarefa que requer precisão de posição, velocidade e
orientação, a fim de garantir a qualidade do revestimento.

A primeira tarefa consiste em cobrir uma superfície no plano $xz$, com
velocidade constante de $40~m/min$ e passo entre paralelos de $3~mm$. A segunda
tarefa é cobrir uma superfície no plano $yz$ com velocidade de $30~m/min$ e
passo de $10~mm$.

Os cartões a seguir resumem as informações referentes às trajetórias de cada
tarefa a ser simulada e as Figuras~\ref{fig::traj1_subf} e \ref{fig::traj2_subf}
apresentam o traçado de cada trajetória no plano correspondente e a
representação no ambiente 3D com o robô no início do primeiro paralelo.
%
\newline
\begin{tcolorbox}
[colframe=black!75!white, colback=white, title = Tarefa 1] 
  \textbf{Área:} $(600 \times 12)~mm^2$ \\
  \textbf{Ponto inicial:} $\mathbf{p}_f = [0,700,~1,238,~-0,600]$ \\
  \textbf{Orientação:} $\boldsymbol{\Phi}_{f} = [0,~0,~0]$ \\
  \textbf{Passo:} $3~mm$ \\
  \textbf{Número de paralelos:} 4 \\
  \textbf{Velocidade da ferramenta:} $40~m/min$
\end{tcolorbox}
%
\begin{figure}[h]
    \centering
    \begin{subfigure}[b]{0.45\textwidth}
        \includegraphics[width=\textwidth]{figs/traj1}
        \caption{Traçado da trajetória 1}
        \label{fig::traj1}
    \end{subfigure}
    \quad %add desired spacing between images, e. g. ~, \quad, \qquad, \hfill
    % etc.
      %(or a blank line to force the subfigure onto a new line)
    \begin{subfigure}[b]{0.49\textwidth}
        \includegraphics[width=\textwidth]{figs/traj1_3d}
        \caption{Trajetória 1 no ambiente 3D}
        \label{fig::traj1_3d}
    \end{subfigure}
    \caption{Trajetória referente à Tarefa 1}
    \label{fig::traj1_subf}
\end{figure}

%
\begin{tcolorbox}
[colframe=black!75!white, colback=white, title = Tarefa 2]
  \textbf{Área:} $(500 \times 40)~mm^2$ \\
  \textbf{Ponto inicial:} $\mathbf{p}_f = [-0,342,~0,900,~-0,300]$ \\
  \textbf{Orientação:} $\boldsymbol{\Phi}_{f} = [0,~\pi/2,~0]$ \\
  \textbf{Passo:} $10~mm$ \\
  \textbf{Número de paralelos:} 4 \\
  \textbf{Velocidade da ferramenta:} $30~m/min$
\end{tcolorbox}
%
\begin{figure}[h!]
    \centering
    \begin{subfigure}[b]{0.45\textwidth}
        \includegraphics[width=\textwidth]{figs/traj2}
        \caption{Traçado da trajetória 2}
        \label{fig::traj2}
    \end{subfigure}
    \quad %add desired spacing between images, e. g. ~, \quad, \qquad, \hfill
    % etc.
      %(or a blank line to force the subfigure onto a new line)
    \begin{subfigure}[b]{0.4\textwidth}
        \includegraphics[width=\textwidth]{figs/traj2_3d}
        \caption{Trajetória 2 no ambiente 3D}
        \label{fig::traj2_3d}
    \end{subfigure}
    \caption{Trajetória referente à Tarefa 2}
    \label{fig::traj2_subf}
\end{figure}


\subsection{Base rígida}

Neste caso, como já foi citado, o modelo MBS utilizado será o desacoplado,
apenas do robô. O sistema tem apenas os 5 graus de liberdade associados às juntas do
manipulador.

Os resultados da simulação para o caso da base rígida é apresentado na
seção~\ref{sec::res_rigida} e serão utilizados como referência para comparação
com os outros modelos de base.


\subsection{Base de testes}

Esta base foi construída para testes do projeto EMMA e consiste em uma
plataforma metálica e modular, com trilho e placa para fixação do robô a uma
altura de aproximadamente $600~mm$ do piso.
A estrutura é formada por perfis leves de alumínio estrutural. O trilho é
composto por perfis paralelos de aço inoxidável e patins de rolamentos o acoplam
à placa de fixação, de alumínio. A Figura~\ref{fig::estrut_modelo_fisico}
apresentou uma fotografia da base sem o robô.

Na seção~\ref{sec::aef} esta base foi utilizada para apresentar o modelo para
Análise por Elementos Finitos, que forneceu sua matriz de rigidez. E na
seção~\ref{sec::experimento} foi detalhado o experimento realizado com o modelo
físico, a fim de determinar os parâmetros para a matriz de amortecimento
proporcional.

O resultado da matriz de rigidez encontrada foi:
%
\begin{equation*}
	K =
\begin{pmatrix}
941537	&	0	&	0	&	0	&	0	&	0 \\
0	&	130543	&	0	&	0	&	0	&	-24981 \\
0	&	0	&	262149	&	0	&	5757	&	0 \\
0	&	0	&	0	&	57019	&	0	&	0 \\
0	&	0	&	5757	&	0	&	119561	&	0 \\
0	&	-24981	&	0	&	0	&	0	&	97595 \\
\end{pmatrix}
\end{equation*}
%
Tal que os valores são expressos em $kN/m$ e $kNm/rad$. 

O resultado experimental forneceu os parâmetros:
%
\begin{align*}
	\alpha &= 8584654 \\
	\beta &= 213145
\end{align*} \todo{Corrigir valores}
%
Que calcula a seguinte matriz de amortecimento:
%
\begin{equation*}
	C =
\begin{pmatrix}
24687 &	0 &	0 &	0 &	0 &	0 \\
0 &	13586 &	0 &	0 &	0 &	-3001 \\
0 &	0 &	22710 &	0 &	795 &	0 \\
0 &	0 &	0 &	5400 &	0 &	0 \\
0 &	0 &	795 & 0 & 3746 & 0 \\
0 &	-3001 &	0 &	0 &	0 &	3265
\end{pmatrix}
\end{equation*}
%
Os resultados das tarefas para esta base são apresentados na
seção~\ref{sec::res_testes}.


\subsection{Base modular PRP}

Esta base, como foi explicado na seção~\ref{sec::insitu}, consiste em um trilho
primário e um trilho secundário sobreposto, que será utilizada para posicionar o
manipulador dentro do ambiente de uma turbina.
Uma plataforma de rotação permite que o trilho secundário forme qualquer ângulo
em relação ao primário. Sobre o trilho secundário corre a placa de fixação do
robô.
Portanto, esta base é classificada como Prismática-Rotacional-Prismática ou PRP,
por fornecer estes graus de liberdade ao posicionamento do robô. 

Na seção~\ref{sec::insitu} apresentou-se esta base no ambiente da turbina.
A Figura~\ref{fig::prp_cad} isola o modelo CAD da base com o robô. Na
configuração que será analisada, o trilho secundário está a $90^\circ$ em
relação ao primário e a placa de fixação do robô a $1100~mm$ de distância da
plataforma de rotação.

\begin{figure}[h]
	\centering 
 	\includegraphics[width=0.80\textwidth]{figs/prp_cad}
 	\caption{Modelo CAD da base modular PRP}
 	\label{fig::prp_cad}
\end{figure}

Para determinação da rigidez, é então modelada esta versão da base para análise
AEF, sendo representada pela Figura~\ref{fig::prp_fea}, com os prinicipais
componentes indicados.

\begin{figure}[h]
	\centering 
 	\includegraphics[width=0.95\textwidth]{figs/prp_fea}
 	\caption{Modelo para AEF da base modular PRP}
 	\label{fig::prp_fea}
\end{figure}

Os perfis, materiais e propriedades de seção do trilho e da estrutura são as
mesmas já descritas no exemplo de aplicação da base de teste. A diferença é a
introdução dos pés da estrutura e da plataforma de rotação. Os pés são formados
por tubos de alumínio de $50~mm$ de diâmetro externo e parede $4~mm$, acoplados
à estrutura por meio de abraçadeiras. O acoplamento será c
As propriedades de seção estão descritas na Figura~\ref{fig::sectrans_pes}. O
material é o Alumínio EN-AW 6060 e suas propriedades foram definidas na
Tabela~\ref{tab::prop_mat}.

\begin{figure}[h]
	\centering 
 	\includegraphics[width=0.80\textwidth]{figs/sectrans_pes}
 	\caption{Propriedades de seção trasnversal dos pés da estrutura}
 	\label{fig::sectrans_pes}
\end{figure}

A plataforma de rotação é formada por duas placas de alumínio, uma solidária ao
trilho primário e a outra ao secundário, unidas por um mancal de rolamento do
tipo \textit{slewing-ring}. Devido a densidade e robustez, esta peça foi
construída com elementos rígidos no modelo para AEF. Logo transmitem forças e
momentos, mas não se deforma. Da mesma forma a placa de fixação foi considerada
rígida e possui um ponto onde virtualmente está acoplado o robô e por onde as
cargas são aplicadas.

As restrições são do tipo engaste e aplicadas às extremidades do pés de apoio da
estrutura. Os deslocamentos prescritos são então aplicados em 6 casos,
exatamente como foi realizado no exemplo da seção~\ref{sec::base}, de acordo com a
Tabela~\ref{tab::casoscarreg}. A matriz de rigidez é então obtida dos resultados
da análise AEF:
%
\begin{equation}
	K = 
\begin{pmatrix}
311723	&	-194	&	-4274	&	-161	&	47454	&	-16 \\
-194	&	15416	&	122	&	-12199	&	-85	&	-2100 \\
-4274	&	122	&	18059	&	1374	&	59	&	5 \\
-161	&	-12199	&	1374	&	13193	&	39	&	1379 \\
47454	&	-85	&	59	&	39	&	18643	&	-7 \\
-16	&	-2100	&	5	&	1379	&	-7	&	18698
\end{pmatrix}
\end{equation}
%
Valores em $kN/m$ e $kNm/rad$.

Para a matriz de amortecimento, seria necessário o ensaio experimental da
estrutura. Como ainda não existe o modelo físico desta vesão da base, os valores
de $\alpha$ e $\beta$ serão estimados. Uma estimativa razoável será utilizar os
mesmos valores obtidos para a base de testes. Portanto, a matriz de
amortecimento torna-se:
%
\begin{equation}
	C = 
\begin{pmatrix}
2150	&	-29	&	-179	&	-8	&	912	&	-4 \\
-29	&	464	&	5	&	-389	&	-12	&	-41 \\
-179	&	5	&	486	&	70	&	-35	&	1 \\
-8	&	-389	&	70	&	446	&	3	&	36 \\
912	&	-12	&	-35	&	3	&	468	&	-2 \\
-4	&	-41	&	1	&	36	&	-2	&	293 \\
\end{pmatrix}
\end{equation}
%
Valores em $kg/s$ e $kg~m/s$. \todo{Corrigir valores}

Os resultados das tarefas simuladas nesta base são apresentados na
seção~\ref{sec::res_prp}.



\subsection{Base com pouca rigidez}

Um último caso de base a ser analisado busca apresentar uma versão muito
flexível, isto é, embora capaz de suportar os esforços com boa margem de
segurança, oferece deslocamentos inaceitáveis para a realização da tarefa.

Na fase de viabilidade técnica do projeto EMMA foi estudada a possibilidade de
utilizar uma estrutura fixada na escotilha superior, que acessa a turbina
próximo ao centro do rotor.
Esta base foi apresentada na seção~\ref{sec::insitu}, na
Figura~\ref{fig::base_telesc_turbina}. Por decisão conservadora, decidiu-se não
utilizá-la como solução, devido a um estudo preliminar de rigidez. Na época
ainda não havia o método proposto neste trabalho e será utilizado agora para
verificar a resposta dinâmica desta base para as tarefas de revestimento.

A base consiste em uma estrutura telescópica, que oferece graus de liberdade do
tipo Prismático-Rotacional-Prismático-Prismático (PRPP), utilizados para
posicionamento do robô.
A Figura~\ref{fig::base_telesc} apresenta modelo CAD da configuração retraída, à
esquerda, e estendida à direita.

\begin{figure}[h]
	\centering 
 	\includegraphics[width=0.65\textwidth]{figs/base_telesc}
 	\caption{Base telescópica PRPP para acesso pela escotilha superior}
 	\label{fig::base_telesc}
\end{figure}

Os itens numerados correspondem à: 
%
\begin{enumerate*}[label=(\arabic*)]
  \item atuador linear tipo sem-fim-coroa;
  \item base fixa na escotilha;
  \item braço prismático \#1;
  \item braço prismático \#2;
  \item atuadores lineares;
  \item junta de rotação;
  \item braço prismático \#3.
\end{enumerate*}
%

O robô é fixado na extremidade do braço prismático \#3. Embora na solução
original fosse utilizado um robô mais leve, será verificada a resposta
deste caso para o manipulador MH12.

Para determinar a matriz de rigidez desta base, é criado seu modelo para AEF.
Será considerada a configuração em que as juntas prismáticas estão totalmente
estendidas e a rotacional fazendo um ângulo de $90^\circ$. A
Figura~\ref{fig::base_telesc_fea} apresenta este modelo indicando os prinicipais
componentes.

\begin{figure}[h]
	\centering 
 	\includegraphics[width=0.45\textwidth]{figs/base_telesc_fea}
 	\caption{Modelo da base telescópica PRPP para AEF}
 	\label{fig::base_telesc_fea}
\end{figure}

Os componentes foram modelados como elementos de viga, tal que é necessário
forecer as propriedades de seção transversal de cada segmento. 
Os braços prismáticos são formados por tubos de seção circular, padrão Schedule.
As propriedades de seção transversal utilizadas na análise são apresentadas na
Tabela~\ref{tab::sectrans_schedules}.
%
\begin{table}[h]
\centering
\caption{Propriedades de seção transversal dos componentes da base PRPP}
\label{tab::sectrans_schedules}
\begin{tabular}{@{}llccc@{}}
\toprule
\textbf{Propriedade}   &           & \textbf{Base fixa} & \textbf{Braços 1 e 2} & \textbf{Braço 3} \\ \midrule
Diâmetro nom.          & [$pol$] & 10                 & 8                     & 6                \\
Padrão Schedule        &           & SCH 40             & SCH40                 & SCH 40           \\
Área de seção          & [$mm^2$]  & 7,68E3             & 5,42E3                & 3,60E3           \\
Inércia polar, J1      & [$mm^4$]  & 1,34E8             & 6,04E7               
& 2,34E7           \\
Momento de Inércia, I2 & [$mm^4$]  & 6,70E7             & 3,02E7                & 1,17E7           \\
Momento de Inércia, I3 & [$mm^4$]  & 6,70E7             & 3,02E7                & 1,17E7           \\
Módulo de seção, S2    & [$mm^3$]  & 4,90E5             & 2,75E5                & 1,39E5           \\
Módulo de seção, S3    & [$mm^3$]  & 4,90E5             & 2,75E5                & 1,39E5           \\ \bottomrule
\end{tabular}
\end{table}
%

Os tubos são de aço ASTM A36, e suas propriedades de material são
apresentadas na Tabela~\ref{tab::astma36}
%
\begin{table}[h]
\centering
\caption{Propriedades de material do aço ASTM A36}
\label{tab::astma36}
\begin{tabular}{@{}llc@{}}
\toprule
\textbf{Propriedade}   & \textbf{}  & \textbf{ASTM A36} \\ \midrule
Densidade              & {[}g/cc{]} & 7,85              \\
Módulo de Elasticidade & {[}GPa{]}  & 200               \\
Coeficiente de Poison  & {[}1{]}    & 0,29              \\
Tensão de Escoamento   & {[}MPa{]}  & 248               \\
Resistência à tração   & {[}MPa{]}  & 400               \\ \bottomrule
\end{tabular}
\end{table}
%

É considerada restrição do tipo engastada no topo da base fixa, representando
sua fixação na entrada da escotilha. Os 6 casos de carregamento e restições são
então aplicados no ponto de fixação do robô, novamente de acordo com a
Tabela~\ref{tab::casoscarreg}.

O resultado para a matriz de rigidez desta base, obtido pelas simulações por
AEF é apresentado na equação~\ref{eq::res_rigidez_prpp}.

%
\begin{equation} \label{eq::res_rigidez_prpp}
	K = 
\begin{pmatrix}
148768	&	-52549	&	0	&	0	&	0	&	10299 \\
-52549	&	57244	&	0	&	0	&	0	&	-21023 \\
0	&	0	&	26634	&	1984	&	12700	&	0 \\
0	&	0	&	1984	&	2926	&	946	&	0 \\
0	&	0	&	12700	&	946	&	9126	&	0 \\
10299	&	-21023	&	0	&	0	&	0	&	11866
\end{pmatrix}
\end{equation}
%
Valores da matriz em $kN/m$ e $kNm/rad$.

Novamente, não é disponível esta versão para ensaio experimental, portanto os
parâmetros $\alpha$ e $\beta$ para formar a matriz de amortecimento são
estimados.
Sejam os valores:
%
\begin{align*}
	\alpha &= 0 \\
	\beta &= 0.00005
\end{align*}
%
A matriz de amortecimento será:
%
\begin{equation} \label{eq::res_amortecimento_prpp}
	C = 
\begin{pmatrix}
7438	&	-2627	&	0	&	0	&	0	&	515 \\
-2627	&	2862	&	0	&	0	&	0	&	-1051 \\
0	&	0	&	1332	&	99	&	635	&	0 \\
0	&	0	&	99	&	146	&	47	&	0 \\
0	&	0	&	635	&	47	&	456	&	0 \\
515	&	-1051	&	0	&	0	&	0	&	593
\end{pmatrix}
\end{equation}
%
Valores da matriz em $kg/s$ e $kg~m/s$.

