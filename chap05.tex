\chapter{Conclusões}

Neste ponto, é importante fazer algumas considerações acerca do método
experimental.
Hoje, existe uma grande variedade de ferramentas CAD e CAE disponíveis,
inclusive gratuitass. Para citar alguns exemplos:
%
\begin{enumerate*}[label=(\roman*)]
  \item \emph{Programas CAD gratuitos}: 123D, LibreCAD, FreeCAD, BRL-CAD,
  OpenSCAD, QCad, SolveSpace
  \item \emph{Programas CAE gratuitos}: CONSELF, FreeFem++, GetFEM++,
  VisualFEA (versão acadêmica).
\end{enumerate*}
%
Por isso, obter as matrizes de inércia e de rigidez podem ser tarefas
relativamente simples, baratas e rápidas de executar, para qualquer estrutura
que se deseje analisar. Além disso, no ambiente virtual, é possível realizar
alterações na estrutura, a fim de se obter os valores desejados de inércia ou
rigidez.

Em comparação, obter a matriz de amortecimento é um processo extremamente
custoso, pelos seguintes motivos:
%
\begin{itemize}
  \item Requer o modelo físico construído;
  \item Requer instrumentos caros como os sensores, martelo e placas de
  aquisição;
  \item Requer cuidado na aquisição, processamento e tratamento dos dados;
  \item Requer \textit{softwares} de processamento e pós-processamento;
  \item Mais difícil de se realizarem modificações estruturais;
  \item Requer boa análise crítica para validar os resultados, o que significa
  ter certo grau de experiência.
\end{itemize}
%

Assim, antes de se realizar o experimento, deve-se julgar qual o impacto do
amortecimento no erro cometido pelo robô. É importante ressaltar que o objetivo
principal do método proposto neste trabalho é oferecer uma forma de estimar se
os erros devido aos movimentos da base inviabilizam a tarefa que o robô deve
cumprir. Para fazer este julgamento, pode-se simplesmente testar, no modelo
teórico acoplado, uma lista de coeficientes para a matriz proporcional de
amortecimento, e verificar os resultados. Se para certa faixa de valores ainda
ocorrem desvios dentro da faixa tolerada do processo, então, a base cumpre com a
rigidez e amortecimento esperados. Em contrapartida, se os valores tolerados
dependem de uma faixa específica de amortecimento, então será válido considerar
a análise experimental.

% -.~.-.~.-.~.-.~.-.~.-.~.-.~.-.~.-.~.-.~.-.~.-

