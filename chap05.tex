\chapter{Conclusões}

% Realce dos achados relevantes e originais ---------------

% A pesquisa oferece um método para verificar se o comportamento dinâmico da base
% de um robô impossibilita a realização adequada de uma tarefa. Novas aplicações
% robóticas, fora do ambiente industrial, os chamados robôs de serviço, requerem a
% instalação de um manipulador robótico para operação \textit{in situ}, muitas
% vezes sendo necessário o projeto de uma base para posicioná-lo, aumentar seu
% alcance ou movimentá-lo em campo. Os requisitos de campo, tais como custo,
% volume, peso, transportabilidade, modularidade e praticidade são incomuns no
% ambiente industrial e representam um desafio conciliá-los com os requisitos
% comuns de um braço robótico projetado para o ambiente bem estruturado. Logo, ser
% capaz de projetar esta base de forma a atender a estes requisitos, é um recurso
% importante e de relevância para o engenheiro projetista.

% Avaliação crítica da própria pesquisa: limitações e aspectos positivos ----

O método de Kane utilizado com o Sophia-Maple, demonstrou uma forma sistemática
de modelar o sistema multicorpos, permitindo a simplificação das equações de
movimento e tornando o algoritmo de solução mais eficiente. A modelagem da
estrutura e simulação pela Análise de Elementos Finitos mostrou-se uma forma
prática e direta de se calcular a rigidez da base para qualquer ponto de
interesse, obtendo-se assim a matriz de rigidez que alimenta o modelo dinâmico
da base.

% Neste ponto, é importante fazer algumas considerações acerca do método
% experimental.
% Hoje, existe uma grande variedade de ferramentas CAD e CAE disponíveis,
% inclusive gratuitas.
% % Alguns exemplos são:
% % % \begin{enumerate*}[label=(\roman*)] \item \emph{Programas CAD gratuitos}:
% % 123D, LibreCAD, FreeCAD, OpenSCAD, QCad; \item \emph{Programas CAE gratuitos}:
% % CONSELF, FreeFem++ e VisualFEA.
% % \end{enumerate*} %
% Assim, obter as matrizes de inércia e de rigidez podem ser uma tarefa
% relativamente simples, econômica e rápida de executar, para qualquer estrutura
% que se deseje analisar. Além disso, no ambiente virtual, é possível realizar
% alterações na estrutura, a fim de se obter os valores da ordem desejada de
% inércia ou rigidez.

Pode-se ressaltar que obter a matriz de amortecimento é um processo custoso,
pelos seguintes motivos:
%
\begin{itemize}
  \item Requer o modelo físico construído;
  \item Requer instrumentos caros como os sensores, martelo e placas de
  aquisição;
  \item Requer cuidado na aquisição, processamento e tratamento dos dados;
  \item Requer \textit{softwares} de processamento e pós-processamento;
  \item Mais difícil de se realizarem modificações estruturais;
  \item Requer boa análise crítica para validar os resultados, o que significa
  ter certo grau de experiência.
\end{itemize}
%

Como esperado, os coeficientes de amortecimento da estrutura metálica da base de
testes, encontrados experimentalmente, apresentaram valores baixos em relação
aos coeficientes de rigidez. Isto implica que as forças dissipativas de
amortecimento são pequenas e influenciam pouco na dinâmica do sistema. Assim,
para estruturas metálicas, sem dispositivos especiais para amortecimento,
pode-se estimar os valores dentro de uma faixa razoável e pequena, para obter
uma solução estimada de simulação e dispensar a necessidade de todo o aparato
experimental e o modelo físico real da estrutura.

% Assim, antes de se realizar o experimento, sugere-se julgar qual o impacto do
% amortecimento na dinâmica do sistema.
% É importante ressaltar que o objetivo principal do método proposto é oferecer
% uma forma de estimar se os erros devido aos movimentos da base inviabilizam a
% tarefa que se deve cumprir. Para fazer este julgamento, pode-se simplesmente
% testar, no modelo teórico acoplado, alguns coeficientes para a matriz
% proporcional de amortecimento, e verificar os resultados. Se o amortecimento é
% crítico para uma certa faixa de valores possível, então será recomendado
% realizar o procedimento experimental para obter o amortecimento efetivo da
% estrutura.


% Comparação crítica com a literatura pertinente

% Interpretação dos achados

Nesta pesquisa foram estudadas 3 bases e 2 tarefas com requisitos distintos de
trajeto, velocidade e orientação da ferramenta acoplada ao efetuador do robô. Os
resultados das simulações demonstraram que a base de testes pode ser considerada
praticamente rígida para realização das tarefas propostas, porque apresentou um
comportamento dinâmico insignificante com relação às variações de posição e
velocidade do ponto de acoplamento base e robô. No entanto, uma pequena
deformação estática da base, após a montagem do robô, propagada até a ponta da
ferramenta acoplada, evidenciou uma notável diferença de trajetória para a
Tarefa 1 e uma pequena mas visível diferença de trajeória para a Tarefa 2.
Ser capaz de prever este aspecto da base, permite que sejam realizadas
alterações estruturais, a fim de evitar a deformação estática excessiva.
A base modular PRP apresentou resultados parecidos, neste sentido, com a base de
testes.

Para a tarefa que foi proposta, de cobertura de revestimento, os erros
sistemáticos notados nas bases de teste e PRP não significam necessariamente a
reprovação de sua utilização, porém para tarefas de outra natureza, como
posicionamento mais preciso, encaixe de peças, operações \textit{pick and
place}, este tipo de erro pode representar a falha da tarefa e reprovação do
projeto.

A base telescópica PRPP apresentou oscilações muito grandes, inclusive
divergindo da trajetória de referência para as duas tarefas. Para o caso
específico de revestimento, o resultado classifica o projeto desta base como
impróprio para a aplicação. Da mesma forma que a base PRP, poderia-se propor
alterações estruturais e verificar-se novos resultados. No entanto, limitações
do acesso pela escotilha restringem o volume desta base, limitando formas mais
variadas de geometria. Se não há alternativa estrutural, deve-se investir em
sistemas de controle mais sofisticados que o PID utilizado, em que o robô pode
até suprimir as oscilações da base com uma estratégia de seguimento de
trajetória e torques de entrada bem elaborada. Como foi citado, existem diversos
métodos de controle voltados para este problema. A vantagem do presente método é
a possibilidade de incorporar estas estratégias de controle ao modelo dinâmico
do Sophia-Maple, com pequenas modificações no código original.

% Conclusão, que pode estar acompanhada de generalização, implicações, perspectivas, recomendações

% Espera-se que que o método possa contribuir na comunidade acadêmica como uma
% ferramenta útil para verificação de estratégias de controle associadas a
% sistemas multicorpos flexíveis. Espera-se ainda que este trabalho contribua para
% facilitar e difundir o uso de robôs para o serviço em campo, trazendo mais
% segurança em diversas operações de alto risco para o ser humano. E também, para
% aumentar a produtividade e eficiência de diversos setores que podem se
% beneficiar de soluções robóticas automatizadas e precisas.



% -.~.-.~.-.~.-.~.-.~.-.~.-.~.-.~.-.~.-.~.-.~.-

