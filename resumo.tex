\begin{abstract}

Apresenta-se neste trabalho um método para prever a falha de um manipulador
robótico para realizar uma tarefa, devido à flexibilidade dos corpos elásticos
que lhe servem como base. Os erros podem ser de posição, velocidade e orientação
da ferramenta na ponta do efetuador. O método também fornece um meio para
verificar como modificações no projeto da base podem alterar o comportamento
dinâmico do sistema acoplado.  A motivação de tal análise se dá pelo crescente
emprego de robôs de serviço para aplicações \textit{in situ}, onde o uso de uma
estrutura leve e esbelta pode ser a melhor e mais viável solução para a
posicionar o robô dentro do ambiente de serviço. A análise se divide em:
modelar pelo método de Kane, os sistemas dinâmicos multicorpos do robô, da base e
do sistema robô e base acoplados; encontrar a matriz de rigidez da base, pela
Análise por Elementos Finitos; encontrar os parâmetros de Rayleigh da matriz de
amortecimento proporcional, por Análise Modal Experimental. Um sistema para
revestimento robótico \textit{in situ} (EMMA) foi escolhido para estudo de caso.
Simulações são realizadas para duas trajetórias e três bases diferentes. Os
resultados do modelo de referência (base rígida) e o modelo acoplado (robô -
base flexível) são então comparados e apresentam as mudanças que ocorrem, devido
à flexibilidade da base, nos parâmetros controlados da tarefa.
Esta análise permite a verificação da conformidade aos requisitos do processo de
revestimento, como posição, velocidade e orientação da pistola com respeito à
superfície.

\end{abstract}

